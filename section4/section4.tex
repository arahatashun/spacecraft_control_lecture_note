\documentclass[class=article, crop=false, dvipdfmx]{standalone}
%% preamble

% titles
\title{宇宙機制御工学}
\date{\the\year/\the\month/\the\day}
\author{Shun Arahata \and Hirotaka Kondo \and Ryosuke Morita}

% packages and libraries
\usepackage[utf8]{inputenc}				%fonts
\usepackage[ipaex]{pxchfon}
\usepackage{pifont}
\usepackage{mathtools, amssymb, mathrsfs, bbm}	%math
\usepackage{siunitx, physics}
\usepackage[table]{xcolor}				%colors
\usepackage{graphicx}					%figures
\usepackage{subcaption, wrapfig}
\usepackage{tikz}
\usetikzlibrary{calc, patterns, decorations, angles, calendar, backgrounds, shadows, mindmap, decorations.pathreplacing}
\usepackage{tcolorbox}					%tables
\usepackage{longtable, float, multirow, array, listliketab, enumitem, tabularx}
\usepackage{listings}					%listings
\usepackage{comment}
\usepackage{hyperref}					%URL, link
\usepackage{url}
\usepackage{pxjahyper}
\usepackage{overcite}					%setting of citation
\usepackage{pxrubrica}					%rubi
\usepackage{fancyhdr, lastpage}			%pagelayout
\usepackage{import, grffile}			%file management
\usepackage{standalone}
\usepackage{bm}
\usepackage{empheq}
% set up for hyperref
\hypersetup{%
	bookmarksnumbered = true,%
	hidelinks,%
	colorlinks = true,%
	linkcolor = black,%
	urlcolor = cyan,%
	citecolor = black,%
	filecolor = magenta,%
	setpagesize = false,%
	pdftitle = {宇宙機制御工学},%
	pdfauthor = {Shun Arahata, Hirotaka Kondo, Ryosuke Morita},%
	pdfkeywords = {},%
	}

% set up for siunitx
\sisetup{%
	%detect-family = true,
	detect-inline-family = math,
	detect-weight = true,
	detect-inline-weight = math,
    %input-product = *,
    quotient-mode = fraction,
	fraction-function = \frac,
	inter-unit-product = \ensuremath{\hspace{-1.5pt}\cdot\hspace{-1.5pt}},
	per-mode = symbol,
	product-units = single,
	}

% setting of line skip
\setlength{\lineskiplimit}{6pt}
\setlength{\lineskip}{6pt}

% setting of indent
\setlength{\parindent}{1zw}
\setlength{\mathindent}{5zw}

% change cite form
\renewcommand{\citeform}[1]{[#1]}

% number equations only when they are referred to in the text
\mathtoolsset{showonlyrefs=true}

% decrement for section
\setcounter{section}{-1}

% circle numbers
\def\maru#1{%
	\leavevmode \setbox0\hbox{$\bigcirc$}%
	\copy0\kern-\wd0 \hbox to\wd0{\hfil{\small#1}\hfil}%
    }

% add explanation to formulas (needs debug)
\newcommand{\overexpl}[3][2pt]{%
	\overset{\hspace{-105mm}\rule[-#1]{0pt}{#1}#3\hspace{-105mm}}{\overbrace{#2}}%
    }
\newcommand{\underexpl}[3][0pt]{%
	\underset{\hspace{-105mm}\rule[0pt]{0pt}{#1}#3\hspace{-105mm}}{\underbrace{#2}}%
    }
\newcommand{\underexplline}[3][0pt]{%
	\underset{\hspace{-105mm}\rule[0pt]{0pt}{#1}#3\hspace{-105mm}}{\underline{#2}}%
    }

% define color of emphasize
\definecolor{midnightblue}{RGB}{25, 105, 112}
\definecolor{mediumblue}{RGB}{0, 0, 165}
\definecolor{royalblue}{RGB}{65, 25, 225}
\definecolor{limegreen}{HTML}{32CD32}
\definecolor{darkgreen}{HTML}{006400}
\definecolor{forestgreen}{HTML}{228B22}
\definecolor{orangered}{HTML}{FF4500}

\definecolor{emph}{HTML}{FF4500}

\begin{document}

\section{Attitude Stabilization}
\renewcommand{\labelitemi}{---}
\begin{itemize}
\item 無制御(CubeSatなど)
\item 受動制御
	\begin{itemize}
    \item G.G.安定
    \item 磁気安定,沿磁力線制御
    \item スピン安定
    	\begin{itemize}
        \item Single Spin
        \item Double Spin
        \end{itemize}
    \end{itemize}
\item 能動制御
	\begin{itemize}
    \item Bias momentum
    \item zero momentum
    \end{itemize}
\end{itemize}
\renewcommand{\labelitemi}{\textbullet}


\subsection{Single Spin Stabilization}

\begin{wrapfigure}{l}{20zw}

\begin{tikzpicture}
\coordinate (O) at (0, 0);

\begin{scope}[-latex, thick]
\draw (O)--(2, -2)node[below right]{$x$};
\draw (O)--(2.4, 2)node[above right]{$y$};
\draw (O)--(0, 3)node[above]{$z$};
\end{scope}

\draw (-2, 1)--(-2, -1);
\draw (2, 1)--(2, -1);
\draw (0, 1) circle [x radius = 2, y radius = 0.6];
\draw (0, -1) circle [x radius = 2, y radius = 0.6];

\end{tikzpicture}
\end{wrapfigure}

$y$をスピン軸,
Principal axisをbody frameとする。\\
外乱トルクがないとき
\begin{align}
I_{xx}\dot{\omega_x}&=(I_{yy}-I_{zz})w_yw_z\\
I_{yy}\dot{\omega_y}&=(I_{zz}-I_{xx})w_zw_x\\
I_{zz}\dot{\omega_z}&=(I_{xx}-I_{yy})w_xw_z
\end{align}
今,$\omega_y=\omega_s(\mbox{目標)}+\epsilon(\mbox{擾乱})\qquad |\epsilon| \ll 1$\\
$|\omega_x|,|\omega_z| \ll 1$と仮定する。

\begin{align}
I_{xx}\dot{\omega_x} &= (I_{yy}-I_{zz})\omega_s\omega_z\\
I_{yy}\dot{\epsilon} &= 0 \\
I_{zz}\dot{\omega_z} &= (I_{xx}-I_{yy})\omega_s
\end{align}
以上の線形の運動方程式になる.


\noindent
\underline{Dynamics解析}
\begin{enumerate}[label=\theenumi)]
\item $y$軸 \qquad $\varepsilon = \mathrm{Const.}$
\item $x, z$軸 \quad Conplug $\rightarrow$ 
	\begin{align}
    I_{xx} \ddot{\omega_x} & = (I_{yy} - I_{zz})\omega_s \dot{\omega_z} \\
    		&= \frac{(I_{yy}-I_{zz})(I_{xx}-I_{yy})}{I_{zz}} {\omega_s}^2 \omega_x
    \end{align}
    ここで
    \[ a = \frac{(I_{yy}-I_{zz})(I_{xx}-I_{yy})}{I_{zz}} {\omega_s}^2 \]
    とおくと,
    \[ \ddot{\omega_x} + a\omega_x = 0 \]
    \[ \ddot{\omega_z} + a\omega_z = 0 \]
\end{enumerate}
安定(発散でない)のための必要条件は
\begin{equation}
a>0\Leftrightarrow I_{yy}>I_{xx},I_{zz}\quad or \quad I_{yy}
< I_{xx}, I_{zz}
\end{equation}
つまりy軸が慣性モーメントの最大軸か最小軸のとき。

\begin{minipage}{0.45\linewidth}

\begin{tikzpicture}[scale=0.6]
\coordinate (O) at (0, 0);

\draw (0, 1) circle [x radius = 3, y radius = 0.8];
\draw (-3, -1) arc [start angle = 180, end angle = 360, x radius = 3, y radius = 0.8];
\draw (3, 1)--+(0, -2);
\draw (-3, 1)--+(0, -2);
\draw [densely dashed] (0, 3)--(0, -2)node[below]{慣性モーメント最大軸};
\draw [->] (0, 3.5)--(0, 3);

\end{tikzpicture}
\end{minipage}
\begin{minipage}{0.3\linewidth}

\begin{tikzpicture}[scale=0.6]
\coordinate (O) at (0, 0);

\draw (0, 2.5) circle [x radius = 1, y radius = 0.5];
\draw (-1, -2.5) arc [start angle = 180, end angle = 360, x radius = 1, y radius = 0.5];
\draw (1, 2.5)--+(0, -5);
\draw (-1, 2.5)--+(0, -5);

\draw [densely dashed] (0, 3)--(0, -3.5)node[below]{慣性モーメント最小軸};
\draw [->] (0, 3.5)--(0, 3);

\end{tikzpicture}
\end{minipage}


\begin{equation}
if\quad a<0 \Rightarrow
\omega_x=
A\exp(\sqrt[]{-a}t)\rightarrow\infty
\end{equation}
$x$と$z$の連成振動\\
簡単のために$I_{xx} = I_{yy}=I_T,I_{yy}I_S$とおく.\\
\begin{align}
\omega_x&=A\sin(\lambda t-\phi)\\
\omega_z&=A\frac{|I_S-I_T|}{I_S-I_T}
\cos(\lambda t-\phi)\\
\lambda&=\sqrt[]{a}=\frac{|I_S-I_T|}{I_T}\omega_s\\
A,\phi&:\text{積分定数}
\end{align}
$\omega_x,\omega_z$は$90^\circ$位相ずれで振動.$\sqrt[]{\omega_x^2+\omega_z^2}=A=\mathrm{Const.}$の円運動\\


\noindent
(body系で見た場合)\\
\begin{center}
\begin{minipage}{0.45\linewidth}

\begin{tikzpicture}[scale=0.8]
\coordinate (O) at (0, 0);

\begin{scope}[-latex]
\draw (0, -2.5)--(0, 2.5)node[right]{$x_b$};
\draw (-2.5, 0)--(2.5, 0)node[right]{$z_b$};
\end{scope}
\draw (O)node[above left = 2mm]{$y_b$} circle [radius = 0.3];

\draw (O) circle [radius = 2];
\draw [->] (120:2)--(122:2);
\draw (60:2) to[bend left = 10] (2.4, 2)node[right]{$\bm{\omega}_b$の軌跡$(I_s > I_T)$};

\begin{scope}[darkgreen]
\draw (O) circle [radius = 1.5];
\draw [->] (40:1.5)--(39:1.5);
\draw (30:1.5) to[bend left = 10] (2.7, 1)node[right]{$\bm{\omega}_b \ (I_s < I_T)$};
\end{scope}

\end{tikzpicture}
\end{minipage}
\hspace{3zw}
\begin{minipage}{0.45\linewidth}

\begin{tikzpicture}
\coordinate (O) at (0, 0);

\begin{scope}[-latex]
\draw (O)--(2, 0)node[right]{$x_b$};
\draw (O)--(0, 2.5)node[left]{$y_b$};
\draw (O)--(-1, -1)node[below left]{$z_b$};
\end{scope}

\draw (0, 1.5) circle [x radius = 1.5, y radius = 0.5];

\draw (O) to[bend left]node[left]{$\omega_y$} (0, 1.5);
\draw [->] (O)--(1.5, 1.5)node[below right]{$\bm{\omega}^b$};
\draw [->] (0.3, 2)--+(-0.01, 0);
\draw (0.5, 1.95) to[bend left=10] (1.5, 2.5)node[right]{$I_s > I_T$のとき};

\end{tikzpicture}

\end{minipage}
\end{center}
\vspace{\baselineskip}

\noindent
(外からみると)\\
\begin{center}

\begin{tikzpicture}
\coordinate (O) at (0, 0);

\begin{scope}[->]
\draw (O)--(2, 1)node[below right]{$I_T \omega_T = \mathrm{Const.}$};
\draw (O)--(-2, 4)node[left]{$I_S \omega_S = \mathrm{Const.}$};
\draw (O)--(0, 5)node[above right]{$\bm{H}$(一定ベクトルat慣性系)};
\draw [red] (O)--(1, 4)node[right]{$\bm{\omega} (I_S > I_T)$};
\draw [darkgreen] (O)--(-1, 3.5)node[above]{$\bm{\omega} (I_S < I_T)$};
\end{scope}

\begin{scope}[densely dashed]
\draw (-2, 4)--(0, 5)--(2, 1);
\draw (2, 1)--(2.5, 1.25)node[above right]{($x,z$軸)\quad$y$に垂直な軸};
\draw (-2, 4)--(-2.5, 5)node[above left]{スピン軸($y$)};
\end{scope}

\draw (0, 1) arc [start angle = 90, end angle = 116, radius = 1];
\node at (80:1.5) {$\theta =\text{Const.}$};

\end{tikzpicture}
\end{center}

\begin{itemize}
\item $\bm{H}$とスピン軸の貼る平面内いつも
$\bm{\omega}$がある。
\item 衛星は$\bm{\omega}$を中心に回る。
\end{itemize}
慣性系においてy(スピン軸)は$\bm{H}$の軸周りに回転する。
これを``Nutation運動''と呼ぶ。

\begin{minipage}{0.45\linewidth}
(1)$I_S<I_T$のとき


\begin{tikzpicture}
\coordinate (O) at (0, 0);

\begin{scope}[->]
\draw (O)--(0, 4)node[right]{$\bm{H}$(fix)};
\draw (O)--(-1.5, 4.5)node[above]{$\bm{\omega}$};
\draw [darkgreen] (O)--(0, 2);
\draw [darkgreen] (O)--(-1, 1);
\end{scope}

\begin{scope}[densely dashed]
\draw (O)--(-3, 3)node[above left]{$y$軸};
\end{scope}

\draw (0, 3) circle [x radius = 1, y radius = 0.3];
\draw (O)--(1, 3)node[right]{space cone};

\draw [rotate = 45] (0, 2.83) circle [x radius = 1.41421356, y radius = 0.3];
\draw (O)--(-3, 1)node[left]{body cone};

\begin{scope}[darkgreen]
\draw (0, 1) to[bend left] (1.5, 1)node[right]{$\frac{H}{I_T}$};
\draw (-0.5, 0.5) to[bend right] (-1, -0.3)node[below]{$\qty(1 - \frac{I_s}{I_T})\omega_s$};
\end{scope}

\end{tikzpicture}
\vspace{6mm}

space coneに外接してbody coneが回転。
\end{minipage}
\hfill
\begin{minipage}{0.45\linewidth}
(2)$I_S>I_T$のとき


\begin{tikzpicture}
\coordinate (O) at (0, 0);

\begin{scope}[->]
\draw (O)--(0, 5)node[right]{$\bm{H}$};
\draw (O)--(1.2, 3.6)node[above]{$\bm{\omega}$};
\draw [darkgreen] (O)--(0, 4);
\draw [darkgreen] (O)--(1, -1);
\end{scope}

\draw (0, 3) circle [x radius = 1, y radius = 0.3];
\draw (O)--(-1, 3);

\begin{scope}[rotate = 30]
\draw [densely dashed] (O)--(0, 4.5)node[above left]{$y$軸};
\draw (0, 2.5) circle [x radius = 2.57, y radius = 1];
\end{scope}
\draw (O)--(-3.2, 0.67)node[below]{body cone};

\begin{scope}[darkgreen]
\draw (0, 1) to[bend left] (1.5, 1)node[right]{$\frac{H}{I_T}$};
\draw (0.5, -0.5) to[bend right] (0, -1)node[below]{$\qty(1 - \frac{I_S}{I_T})\omega_S$};
\draw [darkgreen] (0.1, 3.3) to[bend left](1.5, 4.0)node[right]{space cone};
\end{scope}

\draw [densely dashed] (1.2, 3.6)--(1, -1);

\end{tikzpicture}

space coneに内接しながらbody coneが回転。
\end{minipage}
\vspace{\baselineskip}

いずれの場合も回転速度は$\frac{H}{I_T}$
\\

\noindent
$<$dissipation(エネルギー散逸)\footnote{スロッシング,摩擦,SAP}がある場合$>$\\
$\theta$が一定にならない.
\begin{align}
T&=\frac{1}{2}(I_T{\omega_T}^2+I_S{\omega_S}^2)
\qquad \text{減少する}
\label{T}\\
H^2&=
{I_T}^2+{\omega_T}^2+{I_S}^2+{\omega_S}^2\qquad
\text{外力が働らかなければ一定}
\label{H}\\
\sin\theta&=\frac{I_T \omega_T}{H}\rightarrow
\sin^2\theta=\frac{{I_T}^2{\omega_T}^2}
{H^2}=\frac{I_T}{(I_S-I_T)H^2}(2I_ST-H)
\end{align}
時間微分すると
\begin{align}
2\sin\theta \cos\theta \cdot \dot{\theta}&=
\frac{I_T}{(I_S-I_T)H^2}2I_S\dot{Y}\\
\therefore\qquad \dot{\theta}&=
\frac{2I_SI_T\dot{T}}{\sin2\theta(I_S-I_T)H^2}\\
&=\frac{2I_SI_T\dot{T}}{\sin2\theta(I_S-I_T)H^2}
\end{align}
エネルギー散逸があるので$\dot{T}<0$\\
よって$I_S>I_T$である必要がある。
つまりスピン軸周りの$I$が最大(平らな円筒形)

\begin{center}
\begin{tikzpicture}[scale=0.9]

\coordinate (O) at (0, 0);


\draw (0, 2) circle [x radius = 1, y radius = 0.3];
\draw (-1, -2) arc [start angle = 180, end angle = 360, x radius = 1, y radius = 0.3];
\draw (1, 2)--+(0, -4);
\draw (-1, 2)--+(0, -4);
\draw [densely dashed] (0, 3)--(0, -3);
\draw [->] (-0.5, 2.7) arc [start angle = 180, end angle = 360, x radius = 0.5, y radius = 0.15];

\node at (2.5, 0) {$\Rightarrow$};
\node at (2.5, -0.5) {$\theta$が大きくなる};
\node at (0, 3.0) {if $I_S<I_T$};
\draw (5, 0) circle [x radius = 0.3, y radius = 1];
\draw (9, -1) arc [start angle = -90, end angle = 90, x radius = 0.3, y radius = 1];
\draw (5, 1)--+(4, 0);
\draw (5, -1)--+(4, 0);
\draw [densely dashed] (7, 3)--+(0, -6)node[below]{フラットスピン};

\draw [->] (4, 0) arc [start angle = 180, end angle = 270, radius = 2];
\draw [->] (10, 0) arc [start angle = 0, end angle = 90, radius = 2];

\node at (3, -4) {1958年 エクスプローラ1号};

\end{tikzpicture}

\end{center}


\noindent
・定性的な解釈\\
\[ T=\frac{H^2}{2I_{SS}} \qquad I_{SS}\text{:各瞬間ごとのスピン軸周りの慣性モーメント} \]
($\because T=\frac{1}{2}I_{SS}\omega_{ss}^2$)\\
$H$:一定,$T$が減る。\\
\qquad $T$はどこまで小さくできるか?\\
$\rightarrow$ $I_{SS}$が最大となると$T$最小\\
$\rightarrow$ 最後は$I$最大の軸周りの回転に落ち着く\\
$I_S>I_T$が成り立つとき,$|\dot{T}|$を大きくすれば$|\theta|$は0に戻りやすい。
$\rightarrow$エネルギー散逸を起こしてやればいい。
\begin{enumerate}
\item Nutation damper
\begin{center}

\begin{tikzpicture}
\coordinate (O) at (0, 0);

\draw (O) circle [x radius = 2, y radius = 1];
\draw (O) circle [x radius = 1.8, y radius = 0.8];

\node [right] at (2, 1) {円管の中に粘性が大きい流体};

\end{tikzpicture}
\end{center}

\item Activeな姿勢制御をする
\begin{itemize}
\item スラスタ(Active Nutation Control)
\item MTQ(磁気トルカー)
\end{itemize}
\end{enumerate}



\subsection{Dual Spin Stabilization}
\begin{itemize}
\item 非回転: platform---スピン軸周り$I_{SP}$
\item 回転 : rotar---スピン軸周り$I_{SR}$
\end{itemize}

全体のスピン軸に垂直な軸周り$I_T$


\begin{tikzpicture}

\begin{scope}[rotate=30]
\draw [densely dashed] (0, 5)--(0, -3)node[right]{スピン軸};
\draw (0, 1) circle [x radius = 2.5, y radius = 0.75];
\draw (-2.5, -1) arc [start angle = 180, end angle = 360, x radius = 2.5, y radius = 0.75];
\draw (2.5, 1)--+(0, -2);
\draw (-2.5, 1)--+(0, -2);

\draw [white, fill=white] (1, 3)--(1, 1.5)--(-1, 1.5)--(-1, 3)--cycle;
\draw (0, 3) circle [x radius = 1, y radius = 0.3];
\draw (-1, 1.5) arc [start angle = 180, end angle = 360, x radius = 1, y radius = 0.3];
\draw (1, 3)--+(0, -1.5);
\draw (-1, 3)--+(0, -1.5);

\begin{scope}[->, darkgreen]
\fill (0, -1) circle [radius=0.05];
\draw (0, -1)--(0, 4);
\draw (0, -1)--+(2.5, 0)node[right]{$I_T \omega_T$};
\draw (0, -1)--(2.5, 4)node[right]{$\bm{H}$};
\draw [densely dashed] (2.5, 0)--(2.5, 4)--(0, 4);
\end{scope}
\draw [darkgreen] (0, 0.5) to[bend left=10] (-2.5, 2.5)node[above left]{$I_{SP} \omega_P + I_{SR} \omega_R$};

\begin{scope}[red]
\draw [<-] (1, -0.5)node[right]{$\omega_R$} arc [start angle = 350, end angle = 190, x radius = 1, y radius = 0.3];
\draw [<-] (0.6, 2.2)node[above]{$\omega_R$} arc [start angle = 350, end angle = 190, x radius = 0.6, y radius = 0.18];
\draw (0, 1) arc [start angle = 90, end angle = 40, radius = 1];
\node at (70:1.5) {$\theta$};
\end{scope}
\end{scope}

\end{tikzpicture}






\end{document}
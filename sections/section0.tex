\documentclass[class=article, crop=false, preview=false, dvipdfmx, a4paper]{standalone}
%% preamble

% titles
\title{宇宙機制御工学}
\date{\the\year/\the\month/\the\day}
\author{Shun Arahata \and Hirotaka Kondo \and Ryosuke Morita}

% packages and libraries
\usepackage[utf8]{inputenc}				%fonts
\usepackage[ipaex]{pxchfon}
\usepackage{pifont}
\usepackage{mathtools, amssymb, mathrsfs, bbm}	%math
\usepackage{siunitx, physics}
\usepackage[table]{xcolor}				%colors
\usepackage{graphicx}					%figures
\usepackage{subcaption, wrapfig}
\usepackage{tikz}
\usetikzlibrary{calc, patterns, decorations, angles, calendar, backgrounds, shadows, mindmap}
\usepackage{tcolorbox}					%tables
\usepackage{longtable, float, multirow, array, listliketab, enumitem, tabularx}
\usepackage{listings}					%listings
\usepackage{comment}
\usepackage{hyperref}					%URL, link
\usepackage{url}
\usepackage{pxjahyper}
\usepackage{overcite}					%setting of citation
\usepackage{pxrubrica}					%rubi
\usepackage{fancyhdr, lastpage}			%pagelayout
\usepackage{import, grffile}			%file management
\usepackage{standalone}
\usepackage{bm}
% set up for hyperref
\hypersetup{%
	bookmarksnumbered = true,%
	hidelinks,%
	colorlinks = true,%
	linkcolor = black,%
	urlcolor = cyan,%
	citecolor = black,%
	filecolor = magenta,%
	setpagesize = false,%
	pdftitle = {宇宙機制御工学},%
	pdfauthor = {Shun Arahata, Hirotaka Kondo, Ryosuke Morita},%
	pdfkeywords = {},%
	}

% set up for siunitx
\sisetup{%
	%detect-family = true,
	detect-inline-family = math,
	detect-weight = true,
	detect-inline-weight = math,
    %input-product = *,
    quotient-mode = fraction,
	fraction-function = \frac,
	inter-unit-product = \ensuremath{\hspace{-1.5pt}\cdot\hspace{-1.5pt}},
	per-mode = symbol,
	product-units = single,
	}

% setting of line skip
\setlength{\lineskiplimit}{6pt}
\setlength{\lineskip}{6pt}

% setting of indent
\setlength{\parindent}{1zw}
\setlength{\mathindent}{5zw}

% change cite form
\renewcommand{\citeform}[1]{[#1]}

% number equations only when they are referred to in the text
\mathtoolsset{showonlyrefs=true}

% decrement for section
\setcounter{section}{-1}

% circle numbers
\def\maru#1{%
	\leavevmode \setbox0\hbox{$\bigcirc$}%
	\copy0\kern-\wd0 \hbox to\wd0{\hfil{\small#1}\hfil}%
    }

% add explanation to formulas (needs debug)
\newcommand{\overexpl}[3][2pt]{%
	\overset{\hspace{-105mm}\rule[-#1]{0pt}{#1}#3\hspace{-105mm}}{\overbrace{#2}}%
    }
\newcommand{\underexpl}[3][0pt]{%
	\underset{\hspace{-105mm}\rule[0pt]{0pt}{#1}#3\hspace{-105mm}}{\underbrace{#2}}%
    }
\newcommand{\underexplline}[3][0pt]{%
	\underset{\hspace{-105mm}\rule[0pt]{0pt}{#1}#3\hspace{-105mm}}{\underline{#2}}%
    }

% define color of emphasize
\definecolor{midnightblue}{RGB}{25, 105, 112}
\definecolor{mediumblue}{RGB}{0, 0, 165}
\definecolor{royalblue}{RGB}{65, 25, 225}
\definecolor{limegreen}{HTML}{32CD32}
\definecolor{darkgreen}{HTML}{006400}
\definecolor{forestgreen}{HTML}{228B22}
\definecolor{orangered}{HTML}{FF4500}

\definecolor{emph}{HTML}{FF4500}

\begin{document}

\section{宇宙機の制御の概要}

{\Huge vector font review }
\[ \mathbbm{abcdefghijklmnopqrstuvwxyz} \]
\[ \mathbbm{ABCDEFGHIJKLMNOPQRSTUVWXYZ} \]
\[ \mathbbmtt{abcdefghijklmnopqrstuvwxyz} \]
\[ \mathbbmtt{ABCDEFGHIJKLMNOPQRSTUVWXYZ} \]


宇宙機$\rightarrow$ロケット、人工衛星、宇宙ステーションetc......
\begin{enumerate}[label = \maru{\theenumi}]
\item 航法(Navigation)\\
自分の状態($\rightarrow$位置、速度、角速度、姿勢)を知ること
\item 誘導(Guidance)\\
目標値を決める
\item 制御(Control)\\
目標値を実現するために自分の位置、速度などを変更
\end{enumerate}
\maru{1},\maru{2},\maru{3}全てを合わせて制御と呼ぶこともある.

\begin{enumerate}
      \renewcommand{\labelenumi}{\alph{enumi}).}
      \item 軌道の制御$\cdots$維持(外乱に対して),移行(ランデブー、地球周回からの脱出)
      \item 姿勢の制御$\cdots$カメラのポインティング、変更(姿勢マヌーバ)、エンジンのポインティング
\end{enumerate}
講義では姿勢の制御を扱っていく.


\subsection{姿勢制御の重要な要求}
\begin{enumerate}
\item 姿勢決定精度$\rightarrow$Navigationの精度[deg]\\
		$\mbox{     }\downarrow$\\
 観測対象の位置推定に影響
\item 姿勢安定度$\rightarrow$「ぶれ」の大きさ \quad [deg/sec] or ○ deg 以内(○秒間)
\item 姿勢制御(指向)精度$\rightarrow$特定の方向に向ける精度
\item マニューバビリティ(Maneuverability)$\rightarrow$どれだけ迅速に姿勢変更できるか
\end{enumerate}


\subsection{設計アイテム}
\begin{itemize}
\rubysizeratio{0.8}
\item 機器(センサー(知る)、アクチュエータ(動かす)、Computer(考える))
\item アルゴリズム(決定則、制御則、アノマリー対策)
\item 検証方法(シミュレータ,HILLS$\rightarrow$Hardware-in-the-loop-simulator,3軸テーブル、
軌道上で検証)
\end{itemize}


\subsection{要求精度の例(だいち ALOS-1)}

\begin{tabular}{cll}
\textbullet & 決定精度 & \SI{2e-4}{deg}(オフライン),\ \SI{4e-4}{deg}(オンライン) \\
\textbullet & 安定度 & \SI{4e-4}{deg}を5秒間 \\
\textbullet & 指向精度 & \SI{0.1}{deg}
\end{tabular}

\end{document}
\documentclass[class=article, crop=false, preview=false, dvipdfmx, a4paper]{standalone}
%% preamble

% titles
\title{宇宙機制御工学}
\date{\the\year/\the\month/\the\day}
\author{Shun Arahata \and Hirotaka Kondo \and Ryosuke Morita}

% packages and libraries
\usepackage[utf8]{inputenc}				%fonts
\usepackage[ipaex]{pxchfon}
\usepackage{pifont}
\usepackage{mathtools, amssymb, mathrsfs, bbm}	%math
\usepackage{siunitx, physics}
\usepackage[table]{xcolor}				%colors
\usepackage{graphicx}					%figures
\usepackage{subcaption, wrapfig}
\usepackage{tikz}
\usetikzlibrary{calc, patterns, decorations, angles, calendar, backgrounds, shadows, mindmap, decorations.pathreplacing}
\usepackage{tcolorbox}					%tables
\usepackage{longtable, float, multirow, array, listliketab, enumitem, tabularx}
\usepackage{listings}					%listings
\usepackage{comment}
\usepackage{hyperref}					%URL, link
\usepackage{url}
\usepackage{pxjahyper}
\usepackage{overcite}					%setting of citation
\usepackage{pxrubrica}					%rubi
\usepackage{fancyhdr, lastpage}			%pagelayout
\usepackage{import, grffile}			%file management
\usepackage{standalone}
\usepackage{bm}
\usepackage{empheq}
% set up for hyperref
\hypersetup{%
	bookmarksnumbered = true,%
	hidelinks,%
	colorlinks = true,%
	linkcolor = black,%
	urlcolor = cyan,%
	citecolor = black,%
	filecolor = magenta,%
	setpagesize = false,%
	pdftitle = {宇宙機制御工学},%
	pdfauthor = {Shun Arahata, Hirotaka Kondo, Ryosuke Morita},%
	pdfkeywords = {},%
	}

% set up for siunitx
\sisetup{%
	%detect-family = true,
	detect-inline-family = math,
	detect-weight = true,
	detect-inline-weight = math,
    %input-product = *,
    quotient-mode = fraction,
	fraction-function = \frac,
	inter-unit-product = \ensuremath{\hspace{-1.5pt}\cdot\hspace{-1.5pt}},
	per-mode = symbol,
	product-units = single,
	}

% setting of line skip
\setlength{\lineskiplimit}{6pt}
\setlength{\lineskip}{6pt}

% setting of indent
\setlength{\parindent}{1zw}
\setlength{\mathindent}{5zw}

% change cite form
\renewcommand{\citeform}[1]{[#1]}

% number equations only when they are referred to in the text
\mathtoolsset{showonlyrefs=true}

% decrement for section
\setcounter{section}{-1}

% circle numbers
\def\maru#1{%
	\leavevmode \setbox0\hbox{$\bigcirc$}%
	\copy0\kern-\wd0 \hbox to\wd0{\hfil{\small#1}\hfil}%
    }

% add explanation to formulas (needs debug)
\newcommand{\overexpl}[3][2pt]{%
	\overset{\hspace{-105mm}\rule[-#1]{0pt}{#1}#3\hspace{-105mm}}{\overbrace{#2}}%
    }
\newcommand{\underexpl}[3][0pt]{%
	\underset{\hspace{-105mm}\rule[0pt]{0pt}{#1}#3\hspace{-105mm}}{\underbrace{#2}}%
    }
\newcommand{\underexplline}[3][0pt]{%
	\underset{\hspace{-105mm}\rule[0pt]{0pt}{#1}#3\hspace{-105mm}}{\underline{#2}}%
    }

% define color of emphasize
\definecolor{midnightblue}{RGB}{25, 105, 112}
\definecolor{mediumblue}{RGB}{0, 0, 165}
\definecolor{royalblue}{RGB}{65, 25, 225}
\definecolor{limegreen}{HTML}{32CD32}
\definecolor{darkgreen}{HTML}{006400}
\definecolor{forestgreen}{HTML}{228B22}
\definecolor{orangered}{HTML}{FF4500}

\definecolor{emph}{HTML}{FF4500}

\begin{document}

\section{衛星(Rigid Body)のDynamics}
\begin{center}
%ここになんか入ってたはず
$\xrightarrow[Dynamics]{}$
$\underline{\bm{\omega}}$
$\xrightarrow[Kinematics]{}$
姿勢
\end{center}


\begin{tikzpicture}
\coordinate (O) at (0, 0);
\coordinate (C) at (2, 1);
\coordinate (R) at (2.2, 2);

\begin{scope}[-latex]
\draw (O)--+(-2, -2)node[below left]{$X$}node[above left = 1]{$\bm{I}$};
\draw (O)--+(3, 0)node[right]{$Y$}node[below left = 1]{$\bm{J}$};
\draw (O)--+(0, 3)node[above]{$Z$}node[below left = 1]{$K$};
\end{scope}

\draw [->] (O)--(C)node[right]{$C$};
\draw [->] (O)node[above right=0.8cm]{$\bm{R}$}--(R);
\draw [->] (C)node[above =0.1cm]{$\bm{r}$}--(R);
\draw (0.6, 0.3) to[bend right] (0.8, -0.4)node[below]{$\bm{R}_e$};

\begin{scope}[->]
\draw (C)--+(0.5, 0.25)node[right]{$x(\bm{i})$};
\draw (C)--+(0.3, 0.5)node[right]{$y(\bm{j})$};
\draw (C)--+(-0.3, 0.7)node[left]{$z(\bm{k})$};
\end{scope}

\node at (3, 3) {$P(dm)$};
\end{tikzpicture}

\begin{align}
XYZ\left(\bm{I},\bm{J},\bm{K}\right)&:
\text{慣性座標系}\\
xyz\left(\bm{i},\bm{j},\bm{k}\right)&:
\text{動座標系}\\
\text{C}&:
\text{動座標系の原点}\\
\text{P}&:
\text{微小質量dmの位置}
\end{align}
基本式
\begin{align}
\bm{\dot{x}} = 
\bm{\dot{x}_{rel}} + 
\bm{\omega}\times\bm{x}
\end{align}
$\bm{\dot{x}_{rel}}$は「動座標系から見たx」の変化率\\
質点dmの位置,速度,加速度は
\begin{align}
\bm{R}&=
\bm{R}_c+\bm{r}\\
\bm{V}&=
\dot{\bm{R}}_c+\dot{\bm{r}}\\&=
\bm{V}_c+\bm{V}_{rel}+
\bm{\omega}\times\bm{r}\\
\bm{a}&=
\dot{\bm{V}}_c+\ddot{\bm{V}}_{rel}
+\dot{\bm{\omega}}\times\bm{r}
+%ここもなんか入るはず
\end{align}

\begin{itemize}
\item 以降,成分表記はb-frameから見たものとする.
\item dmの位置$\bm{r}$はb¥f%何これ
\end{itemize}

\begin{tikzpicture}
\coordinate (O) at (0, 0);
\coordinate (C) at (2, 1);
\coordinate (R) at (2.2, 2.5);

\begin{scope}[-latex]
\draw (O)--+(-2, -2)node[below left]{$X$}node[above left=0.8]{$\bm{I}$};
\draw (O)--+(3, 0)node[right]{$Y$}node[below]{$\bm{J}$};
\draw (O)--+(0, 3)node[above]{$Z$}node[below left]{$K$};
\end{scope}

\draw [->] (O)--(C)node[right]{$C$};
\draw [->] (O)node[above right=0.8cm]{$R$}--(R);
\draw (0.6, 0.3) to[bend right] (0.8, -0.4)node[below]{$Re$};

\begin{scope}[->]
\draw (C)--+(0.5, 0.25)node[right]{$x(\bm{i})$};
\draw (C)--+(0.3, 0.5)node[right]{$x(\bm{j})$};
\draw (C)--+(-0.3, 0.7)node[left]{$x(\bm{k})$};
\end{scope}

\node at (3, 3) {$P(dm)$};
\end{tikzpicture}

\begin{align}
XYZ\left(\bm{I},\bm{J},\bm{K}\right)&:
\text{慣性座標系}\\
xyz\left(\bm{i},\bm{j},\bm{k}\right)&:
\text{動座標系}\\
\text{C}&:
\text{動座標系の原点}\\
\text{P}&:
\text{微小質量dmの位置}
\end{align}
基本式
\begin{align}
\bm{\dot{x}} = \bm{\dot{x}_{rel}} + \bm{\omega}\times  \bm{x}
\end{align}
$\bm{\dot{x}_rel}$は「動座標系から見たx」の変化率\\
質点dmの位置,速度,加速度は
\begin{align}
\bm{R}&=\bm{R}_c+\bm{r}\\
\bm{V}&=\dot{\bm{R}}_c+\dot{\bm{r}}\\
&=\bm{V}_c+\bm{V}_{rel}+\bm{\omega}\times
\bm{r}\\
\bm{a}&=
\dot{\bm{V}}_c+\ddot{\bm{V}}_\mathrm{rel}
+\dot{\bm{\omega}}\times\bm{r}
+ \bm{\omega}+\times\dot{\bm{r}}\\
&=
\bm{a}_c+
\left(
\bm{a}_{rel}
+\bm{\omega}\times\bm{V}_{rel}
+\dot{\bm{\omega}}\times
\bm{V}_{rel}
\right)
+\dot{\bm{\omega}}
\times \bm{r}
+\omega\times
\left(
\bm{V}_{rel}
+\bm{\omega}\times\bm{r}
\right)\\
&=
\bm{a}_c+
\bm{a}_{rel}
+\underbrace{2\bm{\omega}\times\bm{V}_{rel}}
_{\text{コリオリ力}}
+\dot{\bm{\omega}}\times\bm{r}
+\underbrace{\bm{\omega}\times
\left( \omega \times \bm{r} \right)}_{\text{遠心力}}
\end{align}


\begin{itemize}
\item 以降,成分表記はb-frameから見たものとする.
\item $dm$の位置$\bm{r}$はb-frameに固定されている
(剛体)$\rightarrow \bm{V}_\mathrm{rel} = \bm{0}, a_\mathrm{rel} = 0$
\end{itemize}


\subsection{諸量の表現}
\begin{itemize}[label = \maru{\theenumi}]
\item{運動量(Linear Momentum}
\begin{align}
\bm{p} \equiv \int_m(\bm{v})dm 
= \int_m(\bm{V}_c + \bm{V}_\mathrm{rel} + \bm{\omega} \times \bm{r})dm \\
= \bm{V}_c \int_m()dm + \omega \times \int_m(\bm{r})dm \\
= \bm{V}_c (\text{M:全体の質量}) \\
 \text{原点を重心に選ぶと}\int_m(\bm{r})dm = \bm{0}
 \end{align}
 \item 角運動量(Angular Momentum)(C点周りの)
 \begin{equation}
 \bm{H}_c = \int_m \bm{r} \times V dm = \int_m \bm{r} times (\bm{V}_c + \bm{\omega} \times \bm{r}) dm
 \end{equation}
ここで,
\begin{equation}
\int_m \bm{r} \times \bm{V}_c dm = \int_m \bm{r} dm \times \bm{V}_c = \bm{0}
\end{equation}
また,
$\bm{r} = \begin{pmatrix} x \\ y \\ z \end{pmatrix}$,
$\bm{\omega} = \begin{pmatrix} \omega_x \\ \omega_y \\ \omega_z \end{pmatrix}$
とおくと,
\begin{equation}
\int_m \bm{r} \times (\bm{\omega} \times \bm{r}) dm = \int_m \bm{r} \times
\begin{pmatrix}
\omega_y z - \omega_z y \\
\omega_z x - \omega_x z \\
\omega_x y - \omega_y x \\
\end{pmatrix}
dm
=
\int_m 
\begin{bmatrix}
\omega_x (y^2 + z^2) - \omega_y xy - \omega_z xz \\
\omega_y (x^2 + z^2) - \omega_z yz - \omega_x xy \\
\omega_z (x^2 + y^2) - \omega_x xz - \omega_y yz
\end{bmatrix}
dm
\end{equation}
\begin{align}
\int_m (x^2 + z^2) dm = I_{yy}, &
\int_m (y^2 + z^2) dm = I_{xx}, &
\int_m (x^2 + y^2) dm = I_{zz}
\end{align}
\begin{align}
\int_m xy dm = |_{xy}, \dots etc
\end{align}
慣性乗積
\begin{align}
\rightarrow
\bm{H}_c^b & =
\begin{bmatrix}
I_{xx} \omega_x - I_{xy}\omega_y - I_{xz}\omega_z \\
I_{yy} \omega_y - I_{xy}\omega_x - I_{yz}\omega_z \\
I_{zz} \omega_z - I_{xz}\omega_x - I_{yz}\omega_y
\end{bmatrix}
=
\begin{bmatrix}
I_{xx} & -I_{xy} & -I_{xz} \\
-I_{xy} & -I_{yy} & -I_{yz} \\
-I_{xz} & -I_{yz} & I_{zz}
\end{bmatrix}
\begin{bmatrix}
\omega_x \\
\omega_y \\
\omega_z
\end{bmatrix} \\
& = [I_\mathrm{ver}] \bm{\omega}^b
\end{align}

\item 力学的エネルギー(kinetic energy)

\begin{align}
T
& =
\frac{1}{2}\int_m \bm{V} \cdot \bm{V} dm
= 
\frac{1}{2} \int_m
\left( \bm{V}_c + \bm{\omega} \times \bm{r} \right)
\left( \bm{V}_c + \bm{\omega} \times \bm{r} \right)\\
& = 
\frac{1}{2} \int_m V_c^2 dm
+ 2 \cdot \frac{1}{2} \bm{V}_c \cdot
\int_m
\underbrace{\bm{\omega} \times \bm{r}}_
{\bm{\omega}\text{を外に出せる}}dm
\\
\end{align}
\begin{align}
=\frac{1}{2}MV_c^2
+ \frac{1}{2}
\end{align}
\begin{align}
\int_m
\left\{
\left(
\omega_{yZ}
-\omega_y
\right)^2
+
\left(
\omega_{zX}
-\omega_y
\right)^2
+
\left(
\omega_{yZ}
-\omega_y
\right)^2
\right
\}
dm
& = 
\frac{1}{2} M {V_c}^2 +
\frac{1}{2}
\left[
I_{xx}{\omega_x}^2 + 
I_{yy}{\omega_y}^2 +
I_{zz}{\omega_z}^2 -
2 \omega_y \omega_z I_{yz}-
2 \omega_z \omega_x I_{xz} -
2 \omega_x \omega_y I_{xy}
\right] \\
&=\frac{1}{2} M{V_c}^2 + \frac{1}{2}
[\omega_x \omega_y \omega_z]{\bm{H}_c}^b
\end{align}

\textcolor{green}{並進$\leftrightarrow$回転}
\end{itemize}


\subsection{Equation of Motion}
\begin{align}
\bm{F} = \bm{\dot{P}}(\text{並進の運動量}) = M\bm{\dot{V}}\\
\bm{M}_c &= \bm{\dot{H}}_c (\text{c点周りの角運動量}) \\
             &= \bm{\dot{H}}_{c,rel} + \omega \times \bm{\dot{H}}_c\\
\rightarrow \bm{M}^b_c = 
[Iner]
\begin{bmatrix}
   \dot{\omega}_x\\
   \dot{\omega}_y\\
   \dot{\omega}_z\\
\end{bmatrix}
+
\begin{bmatrix}
   {\omega}_x\\
   {\omega}_y\\
   {\omega}_z\\
\end{bmatrix}
\times
[Iner]
\begin{bmatrix}
 {\omega}_x\\
   {\omega}_y\\
   {\omega}_z\\
 \end{bmatrix}
\end{align}
x,y,zを慣性主軸に取ると
\begin{equation}
I_{xy} = I_{yz} = I_{zx} = 0
\end{equation}
これより
\begin{equation}
[Iner] = 
\begin{bmatrix}
  I_xx & 0 & 0 \\
  0 & I_yy & 0 \\
  0 & 0 & I_zz 
\end{bmatrix}
\end{equation}


\end{document}
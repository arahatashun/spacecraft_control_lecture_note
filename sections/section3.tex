\documentclass[class=article, crop=false, dvipdfmx, fleqn]{standalone}
%% preamble

% titles
\title{宇宙機制御工学}
\date{\the\year/\the\month/\the\day}
\author{Shun Arahata \and Hirotaka Kondo \and Ryosuke Morita}

% packages and libraries
\usepackage[utf8]{inputenc}				%fonts
\usepackage[ipaex]{pxchfon}
\usepackage{pifont}
\usepackage{mathtools, amssymb, mathrsfs, bbm}	%math
\usepackage{siunitx, physics}
\usepackage[table]{xcolor}				%colors
\usepackage{graphicx}					%figures
\usepackage{subcaption, wrapfig}
\usepackage{tikz}
\usetikzlibrary{calc, patterns, decorations, angles, calendar, backgrounds, shadows, mindmap, decorations.pathreplacing}
\usepackage{tcolorbox}					%tables
\usepackage{longtable, float, multirow, array, listliketab, enumitem, tabularx}
\usepackage{listings}					%listings
\usepackage{comment}
\usepackage{hyperref}					%URL, link
\usepackage{url}
\usepackage{pxjahyper}
\usepackage{overcite}					%setting of citation
\usepackage{pxrubrica}					%rubi
\usepackage{fancyhdr, lastpage}			%pagelayout
\usepackage{import, grffile}			%file management
\usepackage{standalone}
\usepackage{bm}
\usepackage{empheq}
% set up for hyperref
\hypersetup{%
	bookmarksnumbered = true,%
	hidelinks,%
	colorlinks = true,%
	linkcolor = black,%
	urlcolor = cyan,%
	citecolor = black,%
	filecolor = magenta,%
	setpagesize = false,%
	pdftitle = {宇宙機制御工学},%
	pdfauthor = {Shun Arahata, Hirotaka Kondo, Ryosuke Morita},%
	pdfkeywords = {},%
	}

% set up for siunitx
\sisetup{%
	%detect-family = true,
	detect-inline-family = math,
	detect-weight = true,
	detect-inline-weight = math,
    %input-product = *,
    quotient-mode = fraction,
	fraction-function = \frac,
	inter-unit-product = \ensuremath{\hspace{-1.5pt}\cdot\hspace{-1.5pt}},
	per-mode = symbol,
	product-units = single,
	}

% setting of line skip
\setlength{\lineskiplimit}{6pt}
\setlength{\lineskip}{6pt}

% setting of indent
\setlength{\parindent}{1zw}
\setlength{\mathindent}{5zw}

% change cite form
\renewcommand{\citeform}[1]{[#1]}

% number equations only when they are referred to in the text
\mathtoolsset{showonlyrefs=true}

% decrement for section
\setcounter{section}{-1}

% circle numbers
\def\maru#1{%
	\leavevmode \setbox0\hbox{$\bigcirc$}%
	\copy0\kern-\wd0 \hbox to\wd0{\hfil{\small#1}\hfil}%
    }

% add explanation to formulas (needs debug)
\newcommand{\overexpl}[3][2pt]{%
	\overset{\hspace{-105mm}\rule[-#1]{0pt}{#1}#3\hspace{-105mm}}{\overbrace{#2}}%
    }
\newcommand{\underexpl}[3][0pt]{%
	\underset{\hspace{-105mm}\rule[0pt]{0pt}{#1}#3\hspace{-105mm}}{\underbrace{#2}}%
    }
\newcommand{\underexplline}[3][0pt]{%
	\underset{\hspace{-105mm}\rule[0pt]{0pt}{#1}#3\hspace{-105mm}}{\underline{#2}}%
    }

% define color of emphasize
\definecolor{midnightblue}{RGB}{25, 105, 112}
\definecolor{mediumblue}{RGB}{0, 0, 165}
\definecolor{royalblue}{RGB}{65, 25, 225}
\definecolor{limegreen}{HTML}{32CD32}
\definecolor{darkgreen}{HTML}{006400}
\definecolor{forestgreen}{HTML}{228B22}
\definecolor{orangered}{HTML}{FF4500}

\definecolor{emph}{HTML}{FF4500}

\begin{document}
\section{姿勢運動にあたる外乱}
\subsection{Gravity Gradient Torque}
\qquad\qquad --- 地球中心と衛星各部の距離が異なることに起因\\
\begin{minipage}{0.45\linewidth}
\begin{tikzpicture}%ベクトル図
\coordinate (O) at (0, 0);

\begin{scope}[->]
\draw (O)--(-2, 2)node[above]{$dm$}--(-5, -4);
\draw (O)--(-5, -4)node[below left]{地球中心};
\end{scope}

\begin{scope}[-latex, thick, red]
\draw (O)--(1, 1)node[right]{$x$};
\draw (O)--(0.5, -1)node[right]{$y$};
\draw (O)--(-1.5, 0)node[left]{$z$};
\end{scope}

\begin{scope}[-latex, thick, darkgreen]
\draw (O)--(0.2, 1.5)node[right]{$x_0$(速度方向 at 円軌道)};
\draw (O)--(1, -1)node[right]{$y_0$};
\draw (O)--(-1, -0.8)node[below]{$z_0$(地心方向)};
\end{scope}

\draw (O) to[bend left=20] (2, 0)node[right]{衛星中心C};

\node at (-4, 0) {$\bm{R}$};
\node at (-2, -2) {$\bm{R_0}$};
\node at (-1, 1.3) {$\bm{r}$};
\end{tikzpicture}
\end{minipage}\hspace{15zw}
\begin{minipage}{0.4\linewidth}
\begin{tikzpicture}[scale = 0.8]	%孤
\coordinate (O) at (0, 0);

\begin{scope}[-latex]
\draw (O)--(1, 1)node[right]{$x_0$};
\draw (O)--(0, -1)node[below]{$y_0$(面外)};
\draw (O)--(-1, 1)node[right]{$z_0$};
\end{scope}

\draw [->] (O)--(-3, 3)node[above]{地球};
\draw [->] (-3, -1) arc [start angle = -90, end angle = -20, radius = 4.75];
\end{tikzpicture}
\end{minipage}

\noindent
\textcolor{darkgreen}{$x_0, y_0, z_0$ : orbital frame}\\
\textcolor{red}{$x, y, z$ : body frame}\\
$\bm{r}+\bm{R}=\bm{R_0}$\\
dmに働く力$d\bm{F}$は,
\begin{equation}
d\bm{F} = 
\frac{\mu_e \bm{R} dm}{\|\bm{R}\| ^3} =
\mu_e \frac{\bm{R_0} - \bm{r}}{\| \bm{R_0} - \bm{r} \| ^3} dm
\end{equation}
$\bm{r}\ll \bm{R_0}$より
\begin{equation}
d\bm{F} = \frac{\mu_e(\bm{R_0} -\bm{r})}{R_0^3}
\bigl[ 1 + 3\frac{\bm{r} \cdot \bm{R_0}}{R_0^2} + \underbrace{\mathrm{0}(\frac{r^2}{R_0^2})}_{0} \bigr]
\end{equation}
C周りのトルク
\begin{align}
\bm{g} &=
\int_m \bm{r} \times d\bm{F}\\ 
&=\frac{\mu_e}{R_0^3} 
\underbrace{\int \bm{r}\times
(\bm{R}_0 - \bm{r})dm}_{\bm{0}}
+\frac{3\mu_e}{R_0^5} \int \bm{r}\times (\bm{R}_0-\bm{r})
(\bm{r} \cdot \bm{R_0})dm\\
&(\because \bm{r}\times\bm{r}=\bm{0}
,\int \bm{r}dm=\bm{0})
\\
&= \frac{3\mu_e}{R_0^5} \int (\bm{r} \times \bm{R}_0) ( \bm{r} \cdot \bm{R_0} )dm
\end{align}\\
基準座標系として$x_0$,$y_0$,$z_0$は$z_0$が$\bm{R_0}$
方向を向くようにとる.\\
$x_0$,$y_0$,$z_0$から$xyz$(body frame)への変換を,
$z$軸周りに$\psi$,
$y$軸周りに$\theta$,
$x$軸周りに$\phi$の回転で表現する.
\begin{align}
\bm{R_0}^b=R_o
\begin{bmatrix}
-\sin\theta\\
\sin\phi\cos\theta\\
\cos\phi\cos\theta
\end{bmatrix},
\bm{r}^b=
\begin{bmatrix}
x\\
y\\
z
\end{bmatrix}
\end{align}
これ以降,b-frameで見た成分表記とする.
\begin{align}
\int(\bm{r}\times\bm{R}_0)
(\bm{r}\cdot\bm{R}_0)dm
&=R_0^2 \int
\begin{bmatrix}
y\cos\phi\cos\theta  -z\sin\phi\cos\theta\\
-z\sin\theta -x\cos\phi\cos\theta\\
x\sin\phi\cos\theta + y\sin\theta
\end{bmatrix}
(-x\sin\theta+
y\sin\phi\cos\theta+
z\cos\phi\cos\theta)dm\\
&=R_0^2\int
\begin{bmatrix}
(y^2-z^2)
\cos^2\theta\cos\phi\sin\phi
+f_1(xy,yz,zx)\\
(x^2-z^2)\sin\theta\cos\theta\cos\phi
+f_2(xy,yz,zx)\\
(y^2-x^2)\sin\theta\cos\theta\sin\phi+
f_3(xy,yz,zx)\\
\end{bmatrix}dm
\end{align}
\begin{align}
\left\{
\begin{array}{l}
y^2-z^2=(x^2+y^2)-(x^2+z^2)\\
f_1(xy,yz,zx)=f_2(xy,yz,zx)=f_3(xy,yz,zx)=0
(\because\text{慣性主軸})
\end{array}
\right.
\end{align}
\begin{align}
\rightarrow \bm{g}^b = \frac{3\mu_e}{R_0^3}
\begin{bmatrix}
(I_{xx}-I_{yy})\cos^2{\theta}
\cos{\phi} \sin{\phi} \\
(I_{zz}- I_{xx})\sin\theta\cos\theta\cos
\phi\\
(I_{xx}-I_{yy})\sin\theta\cos\theta\sin\phi
\end{bmatrix}
\end{align}
$\theta$と$\phi$が小さいケース
(=z軸がほぼ地心を向いている時)
\begin{align}
\frac{\mu_e}{R_0^3}&=\bm{0}
\end{align}
$\theta$と$\phi$が小さいケース
($= z$軸がほぼ地心を向いているとき)
\begin{align}
\frac{\mu_\mathrm{e}}{{R_0}^3} & = {\omega_0}^2 \qquad (\omega_0 \text{: 軌道角速度}) \\
\bm{g}^\mathrm{b} & \fallingdotseq 3 {\omega_0}^2
\begin{bmatrix}
(I_{zz} - I_{yy}) \phi \\
(I_{zz} - I_{xx}) \theta \\
0
\end{bmatrix}
\end{align}
どういう形がよい?\\
%%%%%%%%%%%%%%
\begin{tikzpicture}%%角度足す
\coordinate (O) at (0, 0);
\coordinate (Y) at (2, -1);
\coordinate (Y0) at (2, 0);
\coordinate (Z) at (-1, -2);
\coordinate (Z0) at (0, -2.2);

\begin{scope}[-latex]
\draw (O)--(Y)node[right]{$y$};
\draw (O)--(Y0)node[right]{$y_0$};
\draw (O)--(Z)node[below]{$z$};
\draw (O)--(Z0)node[below]{$z_0$(地心)};
\end{scope}

\draw [->] (0, -0.5) node[below left = -0.8mm]{$\phi$} arc [start angle = 270, end angle = 243, radius = 0.5];
\draw [->] (0.5, 0) node[below right = -0.8mm]{$\phi$} arc [start angle = 0, end angle = -27, radius = 0.5];

\draw (O) circle [radius = 0.2];
\node [above left = 1] at (O) {$x_0$};
\end{tikzpicture}\\
%%%%%%%%%%%%%%%%%
\begin{minipage}{.4\textwidth}
\begin{tikzpicture}%%	円柱
\coordinate (O) at (0, 0);

\begin{scope}[-latex, thick]
\draw (O)--(2, -2)node[below right]{$x$};
\draw (O)--(2.4, 2)node[above right]{$y$};
\draw (O)--(0, 3)node[above]{$z$};
\end{scope}

\draw (-2, 1)--(-2, -1);
\draw (2, 1)--(2, -1);
\draw (0, 1) circle [x radius = 2, y radius = 0.6];
\draw (0, -1) circle [x radius = 2, y radius = 0.6];
\end{tikzpicture}
%%%%%%%%%%%%%
\end{minipage}
 \hfill
\begin{minipage}{.4\textwidth}
$I_{zz}-I_{yy}>0$\\
$\phi$が正の時トルクも正$\rightarrow$
ずれが拡大$\rightarrow$
不安定
\end{minipage}
%%%%%%%%%%%%%%%%
\begin{minipage}{.4\textwidth}
\begin{tikzpicture}%%円柱
\coordinate (O) at (0, 0);
\begin{scope}[->]
\draw(0, 0.5)--(0, 2);
\draw(0, -0.5)--(0, -2);
\end{scope}

\draw (1, 0.5)--(1, -0.5);
\draw (-1, 0.5)--(-1, -0.5);
\draw (0, 0.5) circle [x radius = 1, y radius = 0.4];
\draw (-1, -0.5) arc [start angle = 180, end angle = 360, x radius = 1, y radius = 0.4];
\draw [densely dashed] (-1, -0.5) arc [start angle = 180, end angle = 0, x radius = 1, y radius = 0.4];
\end{tikzpicture}
\end{minipage}
 \hfill
\begin{minipage}{.4\textwidth}
$I_{zz}-I_{yy}<0$\\
$\phi$が正の時トルクは負$\rightarrow$
ずれが小さくなる$\rightarrow$
安定
\end{minipage}\\
\begin{minipage}{0.4\linewidth}
\begin{tikzpicture}
\coordinate (O) at (0, 0);

\draw (0.8, 1)--(0.8, -1);
\draw (-0.8, 1)--(-0.8, -1);
\draw (0, 1) circle [x radius = 0.8, y radius = 0.3];
\draw (-0.8, -1) arc [start angle = 180, end angle = 360, x radius = 0.8, y radius = 0.3];

\begin{scope}[-latex]
\draw (O)--(1.5, -0.5)node[below]{$x$};
\draw (O)--(1.5, 1)node[right]{$y$};
\draw (O)--(0, -1.8)node[below left]{$z$}node[below right]{地心};
\end{scope}
\end{tikzpicture}
\end{minipage}
\begin{minipage}{0.4\linewidth}
軌道上への「置き方」が重要\\
同様に$\theta$についての安定条件
\end{minipage}
$=I_{zz}-I_{xx}<0$\\
\qquad\qquad$\rightarrow$
「$I_{zz}$が最小」が安定のための必要条件\\
(補足)
\begin{enumerate}
\item gravity gradient torqueは外乱であると同時に,受動的な
姿勢制御に使える($\ang{1}\sim\ang{2}$)
\item 安定といっても$\theta,\phi \rightarrow 0$
に収束しない(0周りに振り子運動する=libration)
\end{enumerate}
$<$姿勢運動(地球指向のため)も考慮に入れたとき$>$\\
\qquad orbital frame 自体が動く(動座標系)\\
基準座標系としてorbital frameをとる($X,Y,Z$)$F_0$
($F_i\cdots$慣性系)\\
$\rightarrow$衛星とともにY軸周りに$\omega_0$で
回転\\
(X,Y,Z)$\rightarrow$xyz(b-frame)への変換を
$\underexpl{F_p}{\text{principle axis}},
(\phi,\theta,\psi)$の微小回転で表現する.\\
\begin{align}
\left\{
\begin{array}{ll}
\phi &= x\text{軸周りの回転}\\
\theta &= y\text{軸周りの回転}\\
\psi &=z\text{軸周りの回転}
\end{array}
\right.
\end{align}
やりたいこと
\begin{equation}
\text{Eulerの式}
\bm{M}^p=
I(\dot{\bm{\omega}}^p)+
\bm{\omega}^p\times
(I\bm{\omega}^p)
\qquad
\text{を書き下したい}
\end{equation}
$\bm{\omega}$ : $F_p$の$F_i$に対する角速度
\begin{equation}
\bm{\omega}
= \underexplline{\bm{\omega}_0}{i \rightarrow O} + \underexplline{\bm{\omega}_\alpha}{O \rightarrow p}\\
\end{equation}
\begin{align}
{\bm{\omega}_o}^{\overexpl[8pt]{i \ \text{or} \ O}{\text{最終的にpで表したいのでDCMが必要}}}
&=
\begin{bmatrix}
0 \\ 
-\omega_C
\\ 0
\end{bmatrix}
\left( \omega_C = \sqrt{\frac{\mu}{R^3}} \right) \\
 {\bm{\omega}_\alpha}^{p} &= 
\begin{bmatrix}
\dot{\phi} \\ \dot{\theta} \\ \dot{\psi}
\end{bmatrix}
\end{align}
$F_0$から$F_p$へのDCMは近似的に,
\begin{equation}
C_0^p = 
\begin{bmatrix}
1 & \psi & -\theta \\
-\psi & 1 & \phi \\
\theta & -\phi & 1
\end{bmatrix}
\qquad \qty(\theta, \ \phi, \ \psi \text{が十分小さい})
\end{equation}
\begin{align}
\rightarrow 
\bm{\omega}^p & =
C_0^p \bm{\omega}_o^o + 
\bm{\omega}_o^\alpha\\
& =
C_0^p 
\begin{bmatrix}
0 \\
-\omega_c \\
0 
\end{bmatrix} + 
\begin{bmatrix} 
\dot{\phi} \\
\dot{\theta} \\ 
\dot{\psi} \\
\end{bmatrix}\\
& = 
\begin{bmatrix}
\dot{\phi} - \omega_c \psi \\
\dot{\theta} - \omega_c \\
\dot{\psi} + \omega_c \phi
\end{bmatrix}\qquad
\left(=
\begin{bmatrix}
\omega_x\\
\omega_y\\
\omega_z
\end{bmatrix}
\right)
\end{align}
Eulerの式に代入すると
\begin{align}
\left\{
\begin{array}{ll}
I_{xx} \dot{\omega_x} 
+ (I_{zz} - I_{yy})\omega_y \omega_z 
&= 3 {\omega_c}^2 (I_{zz} - I_{yz})\phi \\
I_{yy} \dot{\omega_y} 
+ (I_{xx} - I_{zz})\omega_x \omega_z 
&= 3{\omega_c}^2 (I_{zz} - I_{xx}) \theta \\
I_{zz} \dot{\omega_z} 
+ (I_{yy} - I_{zz}) \omega_y \omega_z 
&= 0
\end{array}
\right.
\end{align}
簡略化のために
\begin{equation}
I_{xx}\equiv I_1,
I_{yy}\equiv I_2,
I_{zz}\equiv I_3\quad
\text{とする.}
\end{equation}
(Eulerの式の一つ目)
\begin{align}
&I_1\ddot{\phi}-
I_1\omega_c\dot{\psi}+
(I_3-I_2)(-\omega_c+\dot{\theta})
(\omega_c\phi+\dot{\psi})\\
&=I_1\ddot{\phi}-(I_1+I_3-I_2)
\omega_c\dot{\psi}+
(I_2-I_3)\omega_c-2\phi+
\mathcal{O}(\epsilon^2)\\
&=3\omega_c^2(I_3-I_2)\phi
\end{align}
このようにして外乱がgravity gradientのみの時に
\begin{align}
\begin{cases}
I_1\ddot{\phi}
-(I_1+I_3-I_2)\omega_c\dot{\psi}
+4(I_2-I_3)\omega_c^2
\phi
&=0 \qquad(1)\\
I_2\ddot{\theta}
+3(I_1-I_3)\omega_c^2\theta
&=0 \qquad(2)\\
I_3\ddot{\psi}
+(I_1+I_3-I_2)\omega_c\dot{\phi}
+(I_2-I_1)\omega_c^2\psi
&=0\qquad(3)
\end{cases}
\end{align}
定性的に
\begin{enumerate}
\item y軸まわりは独立している.
$I_1 > I_3$が安定のための必要十分条件
\item x,z軸は$\omega_c$を介してカップリングしている.
(1),(3)より\\
\begin{equation}
A \equiv
\begin{bmatrix}
I_1 & 0\\
0 & I_3 \\
\end{bmatrix}
\qquad
B \equiv
\begin{bmatrix}
0 & -1\\
1 & 0\\
\end{bmatrix}
(I_1 + I_3 - I_2) \omega_c
\qquad
C \equiv
\begin{bmatrix}
4(I_2- I_3) & 0\\
0 & I_2 - I_1\\
\end{bmatrix}
\omega_c
\end{equation}
\begin{equation}
\bm{q} \equiv
\begin{bmatrix}
\phi \\
\psi \\
\end{bmatrix}\qquad
\text{とすると}
\end{equation}
\begin{equation}
A \ddot{\bm{q}} + B \dot{\bm{q}} + C \bm{q} = \bm{0} 
\end{equation}
\end{enumerate}
上式の安定条件は?\\
\quad\underline{十分条件} Cがpositive definite(正定)であること
$(\forall \bm{x}, \bm{x}^T C \bm{x} > 0)$ \\
$\therefore I_2 - I_3 >0\wedge I_2 - I_1 > 0$\\
$\rightarrow$ 1,2から安定のための十分条件は
\begin{equation}
\overbrace{I_2}^{\text{ピッチ}} 
>
\overbrace{I_1}^{\text{ロール}} > 
\overbrace{I_3}^{\text{ヨー}}\qquad
(\omega_c \text{を考えなければ}I_2 > I_3 , I_1 > I_3)
\end{equation}
$<$補足$>$もし厳密な必要十分条件を出すなら
\begin{equation}
k_1=
\frac{I_2-I_3}{I_1},
k_3=
\frac{I_2-I_1}{I_3}
\qquad\text{で運動方程式を書き換えると}
\end{equation}
\begin{equation}
s^4+
(1+3k_1+k_1k_3)\omega_c^2s^2+
4k_1k_3\omega{c}^4=0\qquad
\text{(特性方程式)}
\end{equation}
4つの根が左半面にある条件

\begin{minipage}{0.45\linewidth}
\begin{tikzpicture}
\coordinate (O) at (0, 0);
\draw [-latex] (-2, 0)--(2, 0)node[right]{$k_1$};
\draw [-latex] (0, -2)--(0, 2)node[right]{$k_3$};
\draw (-1.6, -1.6) rectangle (1.6, 1.6);
\draw (2, 2)--(-0.3, -0.3)--(-0.3, -1.9)--(0, -1.6);
\fill [pattern = north west lines] (O)--(1.6, 1.6)--(1.6, 0)--cycle;
\fill [pattern = north west lines] (O)--(-0.3, -0.3)--(-0.3, -1.9)--(0, -1.6)--cycle;

\draw [darkgreen] (1, 0.5) to[bend left=20] (2.2, 1.2)node[right]{$I_2 > I_1 > I_3$に対応};
\draw [darkgreen] (-0.15, -1) to[bend left=20] (0.5, -2.4)node[right]{$I_2 > I_1 > I_3$にはない領域(エネルギー差につがあると不安定になる)};
\end{tikzpicture}
\end{minipage}
\begin{minipage}{0.45\linewidth}
\begin{itemize}
\item $1+3k_1+k_1k_3>0$
\item $k_1k_3>0$
\item $
(1+3k_1+k_1k_3)^2-
16k_1k_3>0
$
\end{itemize}
\end{minipage}



\section{Solar Radiation Pressure Torque}
太陽からくるphoton(光子)の力

\begin{tikzpicture}
\draw [<-] (-3, 0)node[left]{$\bm{s}$}--(3, 0);
\draw (-3, 3)--(3, -3);
\draw [<-] (-2.5, -2)node[below left]{$\bm{n}$}--(2.5, 2);
\draw [scale=0.8] (-3.3, 2.7)--(-2.7, 3.3)--(3.3, -2.7)--(2.7, -3.3)--cycle;
\draw [->] (0, 0)--(0.6, 2.5)[above]node{鏡面反射};
\draw [<-] (4, 1)--+(1, 0);
\draw [<-] (4, 0)--+(1, 0)node[right]{Sun}node[below=12pt,  right]{$P$(輻射圧〔\si{N/m^2}〕)};
\draw [<-] (4, -1)--+(1, 0);

\draw (0.8, 0) arc [start angle = 0, end angle = 78, radius = 0.8cm];
\node at (58:1.2) {$\psi$};
\node at (20:1.2) {$\psi$};
\draw (2, -2.1) to[bend right] (1, -2)node[left]{面積$A$};
\end{tikzpicture}

3通りの方法
\begin{enumerate}[label = (\theenumi)]
\item 吸収分(absorption)\quad
$\bm{F}_a=\underbrace{\rho_a}_{\text{吸収する割合}}P\underbrace{(\bm{n}\cdot\bm{s})}_{\textrm{正面断面積}}\bm{s}$
\item 鏡面反射(specular reflection)\quad
$\bm{F}_{s}=\underline{2}\rho_sPA(\bm{n}\cdot\bm{s})^2\bm{n}$
\item 散乱反射(拡散反射) (diffuse reflection)\quad
$\bm{F}_d=\rho_dPA(\bm{n}\cdot\bm{s})\bm{s}+\frac{2}{3}\rho_dPA(\bm{n}\cdot\bm{s})\bm{n}$\\
法線方向とのなす角のcosに比例して反射する.

\end{enumerate}

\begin{tikzpicture}[scale = 0.7]
\coordinate (O) at (0, 0);
\draw (-2, 0)--(2, 0);
\draw (0, 2) circle [radius = 2cm];
\begin{scope}[->]
\draw (O)--(0, 4)node[above left]{$\bm{n}$};
\draw (O)--(0.5, 3.9);
\draw (O)--(1.7, 3);
\draw (O)--(2, 2);
\draw (O)--(-0.5, 3.9);
\draw (O)--(-1.7, 3);
\draw (O)--(-2, 2);
\end{scope}
\end{tikzpicture}

(1)~(3)を足し合わせると,
\begin{equation}
\bm{F} = \bm{F}_a + \bm{F}_s + \bm{F}_d
	= PA(\bm{n}\cdot \bm{s})\qty{(\rho_a + \rho_d)\bm{s} + \qty(2 (\bm{n}\cdot\bm{s})\rho_s + \frac{3}{2} P_d)\bm{n}}
\end{equation}
ただし,$\rho_a + \rho_d + \rho_s = 1$(透過しない材料を仮定)
トルクにする場合
\[ \bm{M} = \sum_i \bm{r}_i \times \bm{F}_i \]
$\bm{r}_i$:(Gから各面の圧力中心までの位置ベクトル)(面の形状中心が多い)

\noindent
(補足)
\[ P = \SI{4.6e-6}{N/m^2} \quad (\text{\SI{1}{km^2}で\SI{4.6}{N}}) \]
(例題)地球指向衛星の太陽電池パドルに当たった場合\\
\begin{itemize}
\item $X_0,Y_0,Z_0$:
基準のorbital frame($\bm{I}_0\bm{J}_0,\bm{K}_0$)
\item $X,Y,Z$:
orbital frame($\bm{I},\bm{J},\bm{K}$)
\end{itemize}

\begin{tikzpicture}
\coordinate (O) at (0, 0);
\coordinate (A) at (-2.7, 1.97);

\draw [densely dashed] (-3, 0)node{$\times$}node[below]{地球中心}--(3, 0)node[right]{軌道面};

\draw (3, 2)node[below=4mm]{太陽} circle [radius = 2mm];
\draw [decorate, decoration={ticks, segment length = 2.36mm, amplitude = 0.5mm}] (3, 2) circle [radius = 3mm];

\draw [->] (2.45, 1.75)--(0.9, 0.6);
\draw (1.2, 0.8) node[right=1mm]{$\beta$} arc [start angle = 30, end angle = 0, radius = 0.3];

\draw (-3, 2) arc [start angle = 90, end angle = -90, x radius = 3, y radius = 2];

\draw [->] (-2.7, 0)node[above right]{$\alpha$} arc [start angle = 0, end angle = 80, radius = 0.3];
\draw (-3, 0)--(A);

\draw (0.3, 0.3) rectangle (-0.3, -0.3);
\draw (0, 0.3)--+(0, 0.3);
\draw (0, -0.3)--+(0, -0.3);
\draw (-0.3, 0.5) rectangle (0.3, 1.5);
\draw (-0.3, -0.5) rectangle (0.3, -1.5);

\begin{scope}[darkgreen, -latex]
\draw (0, 0)--(0, 1)node[above=2mm, right=-3mm]{$X_a$(進行方向)};
\draw (0, 0)--(0, -1)node[below=2mm, right=-3mm]{$Y_a$(面外方向)};
\draw (0, 0)--(-1, 0)node[above]{$Z_a$};
\end{scope}

\draw ($(A)+(0.3, 0.3)$) rectangle ($(A)+(-0.3, -0.3)$);
\draw ($(A)+(0, 0.3)$)--+(0, 0.3);
\draw ($(A)+(0, -0.3)$)--+(0, -0.3);
\draw ($(A)+(-0.3, 0.5)$) rectangle ($(A)+(0.3, 1.5)$);
\draw ($(A)+(-0.3, -0.5)$) rectangle ($(A)+(0.3, -1.5)$);

\begin{scope}[red, -latex]
\draw (A)--+(-1, 0.2)node[left]{$X$};
\draw (A)--+(0, -1)node[below, right=-1mm]{$Y$};
\draw (A)--+(-0.15, -1)node[left=1mm]{$Z$};
\end{scope}
\end{tikzpicture}

$\beta$:軌道面と太陽方向のなす角
\begin{equation}
\bm{s}=\sin\beta\bm{J}_0+\cos\beta\bm{K}_0
\end{equation}
これをXYZ軸で表すと
\begin{align}
\bm{J}_0=\bm{J},\bm{K}_0=\bm{K}\cos\alpha
+\bm{I}\sin\alpha\\
\therefore \bm{s}=\sin\beta\bm{J}+
\cos\beta\cos\alpha\bm{K}+
\cos\beta\sin\alpha\bm{I}\\
\therefore
\bm{s}^0=
\begin{bmatrix}
\cos\beta\sin\alpha\\
\sin\beta\\
\cos\beta\cos\alpha
\end{bmatrix}
\end{align}

・太陽電池パドルの法線方向$\bm{n}$は?
$\rightarrow$軸触りのジンバルによって$-z_0$方向に指向し続ける。
\begin{align}
\bm{n} & = \bm{K}_0 = \bm{K} \cos\alpha + \bm{I} \sin\alpha \\
\rightarrow \bm{n}^0 & = \begin{bmatrix} \sin\alpha \\ 0 \\ \cos\alpha
\end{bmatrix}
\end{align}
圧力中心のbody frame (ordinal frame)での位置を$\begin{bmatrix} x \\ y \\ z \end{bmatrix}$とすると,
\begin{align}
\bm{M}_i & = PA \underexpl{\cos\beta}{(\bm{n}\cdot\bm{s})} \begin{bmatrix} x \\ y \\ z \end{bmatrix} \times
\begin{bmatrix}
(1-\rho_s)\sin\alpha\cos\beta + 2\qty(\cos\beta \rho_s + \frac{1}{3} \rho_d)\sin\alpha \\
(1-\rho_s) \sin\beta
(1-\rho_s)\cos\alpha \cos\beta + \qty(\cos\beta \rho_s \frac{1}{3} \rho_d) \cos\alpha
\end{bmatrix}
\end{align}
ここで,
\begin{equation}
\left\{ \begin{array}{l}
K_1 = (1-\rho_s) \cos\beta + 2\qty(\cos\beta \rho_s \frac{1}{3} \rho_d) \\
K_ = (1-\rho_s) \sin\beta
\end{array} \right.
\end{equation}
を導入する($\beta$は季節変動するが,定数と見る)。
\begin{align}
\therefore
\bm{M}_a=PA\cos\beta\begin{bmatrix}
x\\
y\\
z\\
\end{bmatrix}
\times\begin{bmatrix}
K_1\cos\beta\\
K_2\\
K_1\cos\alpha\\
\end{bmatrix}
=
PA\cos\beta\begin{bmatrix}
yK_1\cos\alpha-\underline{zK_2}\\
zK_1\sin\alpha-xK_1\cos\alpha\\
\underline{xK_2}-yK_1\sin\alpha
\end{bmatrix}
\end{align}
定性的に見ると
\begin{itemize}
\item $\cos\alpha,\sin\alpha\rightarrow$
一周で積分すると0
\item $zK_2,xK_2\rightarrow$角運動量が蓄積される
\end{itemize}
よって
\begin{enumerate}
\item $K_2=0$にするには,$\rho_s=1\rightarrow $NG(発電できない)
\item $x=z=0$にするには対称なパドルにすれば良い(+Y片翼でもいい)
\end{enumerate}

\begin{align}
P & = \SI{4.6e-6}{N/m^2} \quad \text{at \SI{1}{AU}} \\
	& = \SI{3e-5}{N\m^2} \quad \text{at Mercury} \\
    & = \SI{3e-9}{N/m^2} \quad \text{at Pluto}
\end{align}
地球表面での太陽光の反射 ~34\% (モデル作りにくい,大気の揺らぎなどの影響を受ける)
\[ \SI{2e-6}{N/m^2} \quad \text{(LEO)} \]
\[ \SI{3e-8}{N/m^2} \quad \text{(GEO)} \]
惑星自身のradiation(\underline{赤外}輻射)\\
$\quad $地球:アルバドの$\frac{1}{3}$程度,木星などはradiationが大きい(反射と比較して大きい)
\subsection{Magnetic Torque}
\begin{align}
\bm{g}_m&=\bm{m}_m\times\bm{B}\\
\bm{m}_n&:\textrm{magnetic moment of space craft}
\begin{cases}
\text{永久磁石}\\
\text{回路(電流ループ)}\\
\text{磁化した金属}
\end{cases}\\
\bm{B}&:\textrm{external geomangetic flux density}
\end{align}

\begin{tikzpicture}
\coordinate (O) at (0, 0);
\draw (O) circle [radius = 2cm];
\draw (0, 2)--+(0, 0.3);
\draw (0, -2)--+(0, -0.3);
\draw (-2, 0) arc [start angle = 180, end angle = 360, x radius = 2cm, y radius = 0.8cm];
\draw (1.8, -0.3) to[bend right=20] (2.2, -0.8)node[right]{赤道};

\begin{scope}[darkgreen]
\draw [rotate = 30] (O) circle [x radius = 2cm, y radius = 0.8cm];
\draw (1.7, 0.6) to[bend right=20] (2.2, 0.8)node[right]{磁気の赤道};
\end{scope}

\begin{scope}[red]
\draw [-latex, rotate=30] (O)--(-2.4, 0)node[below]{$X$};
\draw [-latex, rotate=30] (O)--(0, 2.4)node[above left]{$Z$};
\draw (-1.1, 1.9) to[bend left=20] (1.8, 2)node[right]{磁北極(\ang{78.5}N, \ang{69}W)};
\end{scope}

\draw (O)--(-2, 0);
\draw (-0.4, 0) arc [start angle = 180, end angle = 210, radius = 0.4];
\draw (-0.5, 0) arc [start angle = 180, end angle = 210, radius = 0.5];
\draw (195:0.5) to[bend left=10] (-2, 0.8)node[left]{$\lambda_{m}$};
\end{tikzpicture}

地磁気のダイポールモーメント(ポテンシャル場)
\begin{equation}
\phi_m=-\frac{\mu_m}{R^2}\sin\lambda_m
\end{equation}
\begin{itemize}
\item $\mu_m$:
$1\times10^7$Wb$\cdot$m
\item  R:
地球中心からの距離
\item $\lambda_m$:
geomagnetic equational planeから測ったlatitude
\end{itemize}




\begin{equation}
\rightarrow \bm{B}^m = - nabra \phi_m
= - \frac{\mu_m}{R^3}
\begin{bmatrix}
3 \sin\lambda_m \cos\lambda_m \cos\eta_{\overexpl{n}{経度}} \\
3 \sin\lambda_m \cos\lambda_m \sin\eta_n \\
3 \sin^2\lambda_m - 1
\end{bmatrix}
\end{equation}
衛星に働くトルクを推定するには,
\begin{equation}
\bm{q}_m^b = \bm{m}_m^b \times 
%\qty(
%\underexpl{C_m^b}{\underexplline{C_i^b}{$\uparrow$姿勢データ} \underexplline{C_m^i}{Const.}} 
%\underexplline{B^m}{\text{データベース(IGRF)}})
\end{equation}
$\bm{m}_m^b$をどうやって推定するか?
\begin{enumerate}[label = \maru{\theenumi}]
\item 地上で計測(磁気シールド室)\\
	\quad 様々な動作モードでの計測が必要
\item 軌道上で推定する:カルマンフィルターなど$\rightarrow$ $-m_m^b$の磁気モーメントでキャンセルすることもある
\end{enumerate}



\subsection{その他外乱トルク}
\begin{enumerate}[label = \maru{\theenumi}]
\item 粒子との衝突 meteoroid debris(\~{}8km/s, Jspocが監視してデータベース化)
\item Thermoelastc deformation
\begin{align}
\begin{cases}
\text{CGの位置
\text{慣性モーメントの変化}}
\end{cases}
\end{align}
夜から昼に出た瞬間に温度差大"heat shock" サーマル
スナップ
\item 非環境的トルク
(内部トルクは角運動量を変えない)
\begin{itemize}
\item flexibility
\item crew motion (ISS)
\item fuel sloshing
\item Rwじょう乱
\end{itemize}
\end{enumerate}
\end{document}
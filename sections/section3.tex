\documentclass[class=article, crop=false, preview=false, dvipdfmx, a4paper]{standalone}
%% preamble

% titles
\title{宇宙機制御工学}
\date{\the\year/\the\month/\the\day}
\author{Shun Arahata \and Hirotaka Kondo \and Ryosuke Morita}

% packages and libraries
\usepackage[utf8]{inputenc}				%fonts
\usepackage[ipaex]{pxchfon}
\usepackage{pifont}
\usepackage{mathtools, amssymb, mathrsfs, bbm}	%math
\usepackage{siunitx, physics}
\usepackage[table]{xcolor}				%colors
\usepackage{graphicx}					%figures
\usepackage{subcaption, wrapfig}
\usepackage{tikz}
\usetikzlibrary{calc, patterns, decorations, angles, calendar, backgrounds, shadows, mindmap}
\usepackage{tcolorbox}					%tables
\usepackage{longtable, float, multirow, array, listliketab, enumitem, tabularx}
\usepackage{listings}					%listings
\usepackage{comment}
\usepackage{hyperref}					%URL, link
\usepackage{url}
\usepackage{pxjahyper}
\usepackage{overcite}					%setting of citation
\usepackage{pxrubrica}					%rubi
\usepackage{fancyhdr, lastpage}			%pagelayout
\usepackage{import, grffile}			%file management
\usepackage{standalone}
\usepackage{bm}
% set up for hyperref
\hypersetup{%
	bookmarksnumbered = true,%
	hidelinks,%
	colorlinks = true,%
	linkcolor = black,%
	urlcolor = cyan,%
	citecolor = black,%
	filecolor = magenta,%
	setpagesize = false,%
	pdftitle = {宇宙機制御工学},%
	pdfauthor = {Shun Arahata, Hirotaka Kondo, Ryosuke Morita},%
	pdfkeywords = {},%
	}

% set up for siunitx
\sisetup{%
	%detect-family = true,
	detect-inline-family = math,
	detect-weight = true,
	detect-inline-weight = math,
    %input-product = *,
    quotient-mode = fraction,
	fraction-function = \frac,
	inter-unit-product = \ensuremath{\hspace{-1.5pt}\cdot\hspace{-1.5pt}},
	per-mode = symbol,
	product-units = single,
	}

% setting of line skip
\setlength{\lineskiplimit}{6pt}
\setlength{\lineskip}{6pt}

% setting of indent
\setlength{\parindent}{1zw}
\setlength{\mathindent}{5zw}

% change cite form
\renewcommand{\citeform}[1]{[#1]}

% number equations only when they are referred to in the text
\mathtoolsset{showonlyrefs=true}

% decrement for section
\setcounter{section}{-1}

% circle numbers
\def\maru#1{%
	\leavevmode \setbox0\hbox{$\bigcirc$}%
	\copy0\kern-\wd0 \hbox to\wd0{\hfil{\small#1}\hfil}%
    }

% add explanation to formulas (needs debug)
\newcommand{\overexpl}[3][2pt]{%
	\overset{\hspace{-105mm}\rule[-#1]{0pt}{#1}#3\hspace{-105mm}}{\overbrace{#2}}%
    }
\newcommand{\underexpl}[3][0pt]{%
	\underset{\hspace{-105mm}\rule[0pt]{0pt}{#1}#3\hspace{-105mm}}{\underbrace{#2}}%
    }
\newcommand{\underexplline}[3][0pt]{%
	\underset{\hspace{-105mm}\rule[0pt]{0pt}{#1}#3\hspace{-105mm}}{\underline{#2}}%
    }

% define color of emphasize
\definecolor{midnightblue}{RGB}{25, 105, 112}
\definecolor{mediumblue}{RGB}{0, 0, 165}
\definecolor{royalblue}{RGB}{65, 25, 225}
\definecolor{limegreen}{HTML}{32CD32}
\definecolor{darkgreen}{HTML}{006400}
\definecolor{forestgreen}{HTML}{228B22}
\definecolor{orangered}{HTML}{FF4500}

\definecolor{emph}{HTML}{FF4500}

\begin{document}

\section{姿勢運動にあたる外乱}
\subsection{Gravity Gradient Torque}
\qquad\qquad --- 地球中心と衛星各部の距離が異なることに起因

\begin{tikzpicture}%ベクトル図
\coordinate (O) at (0, 0);

\begin{scope}[->]
\draw (O)--(-2, 2)node[above]{$dm$}--(-5, -4);
\draw (O)--(-5, -4)node[below left]{地心};
\end{scope}

\begin{scope}[-latex, thick, red]
\draw (O)--(1, 1)node[right]{$x$};
\draw (O)--(0.5, -1)node[right]{$y$};
\draw (O)--(-1.5, 0)node[left]{$z$};
\end{scope}

\begin{scope}[-latex, thick, green]
\draw (O)--(0.2, 1.5)node[right]{$x_0$(速度方向の円軌道)};
\draw (O)--(1, -1)node[right]{$y_0$};
\draw (O)--(-1, -1)node[below]{$z_0$(地心方向)};
\end{scope}

\draw (O) to[bend left=20] (2, 0)node[right]{衛星中心C};

\node at (-4, 0) {$\bm{R}$};
\node at (-2, -2) {$\bm{R_0}$};
\node at (-1, 1.3) {$\bm{r}$};
\end{tikzpicture}

\begin{tikzpicture}%孤
\coordinate (O) at (0, 0);


\begin{scope}[-latex]
\draw (O)--(1, 1)node[right]{$x_0$};
\draw (O)--(0, -1)node[below]{$y_0$(面外)};
\draw (O)--(-1, 1)node[right]{$z_0$};
\end{scope}

\draw [->] (O)--(-3, 3)node[above]{地球};
\draw [->] (-3, -1) arc [start angle = -90, end angle = -20, radius = 4.75];
\end{tikzpicture}


\textcolor{green}{$x_0, y_0, z_0$ : orbital frame}\\
\textcolor{red}{$x, y, z$ : body frame}

dmに働く力$d\bm{F}$ は,
\begin{equation}
d\bm{F} = 
\frac{\mu_e \bm{R} dm}{\|\bm{R}\| ^3} =
\mu_e \frac{\bm{R_0} - \bm{r}}{\| \bm{R_0} - \bm{r} \| ^3} dm
\end{equation}
$\bm{r}\ll \bm{R_0}$より
\begin{equation}
d\bm{F} = \frac{\mu_e(\bm{R_0} -\bm{r})}{R_0^3}
\bigl[ 1 + 3\frac{\bm{r} \cdot \bm{R_0}}{R_0^2} + \mathrm{0}(\frac{r^2}{R_0^2}) \bigr]
\end{equation}
C周りのトルク
\begin{align}
\bm{g} =
\int_m \bm{r} \times d\bm{f} 
&=\frac{\mu_e}{R_0^3} \int \bm{r}\times (\bm{R_0} - \bm{r})dm\\
&+\frac{3\mu_e}{R_0^5} \int (\bm{r}\times \bm{R_0}) 
(\bm{r} \cdot \bm{R_0})dm\\
&= \frac{3\mu_e}{R_0^5} \int (\bm{r} \times \bm{R_0}) ( \bm{r} \cdot \bm{R_0} )dm
\end{align}
\begin{align}
\frac{\mu_e}{R_0^3} \int \bm{r}\times (\bm{R_0} - \bm{r})dm = 0\\
\because \bm {r} \times \bm{r} = 0\\
\int \bm{r} dm = \bm{0}
\end{align}
基準座標系として$x_0$,$y_0$,$z_0$は$z_0$が$\bm{R_0}$
方向を向くようにとる.\\
$x_0$,$y_0$,$z_0$から$xyz$(body frame)への変換を,
$z$軸周りに$\psi$,
$y$軸周りに$\theta$,
$x$軸周りに$\phi$の回転で表現する.
\begin{align}
\bm{R_0}^b=R_o
\begin{bmatrix}
-\sin\theta\\
\sin\phi\cos\theta\\
\cos\phi\cos\theta
\end{bmatrix},
\bm{r}^b=
\begin{bmatrix}
x\\
y\\
z
\end{bmatrix}
\end{align}
これ以降,b-frameで見た成分表記とする.

\begin{align}
\rightarrow \bm{g}^b = \frac{3\mu_e}{R_0^3}
\begin{pmatrix}
(I_{xx}-I_{yy})\cos{\theta}^2 \cos{\phi} \sin{phi}^2 \\
(I_{zz}- I_{}
\end{pmatrix}
\end{align}
$\theta$と$\phi$が小さいケース
(=z軸がほぼ地心を向いている時)
\begin{align}
\frac{\mu_e}{R_0^3}&=\bm{0}
\end{align}







$\theta$と$\phi$が小さいケース
($= z$軸がほぼ地心を向いているとき)
\begin{align}
\frac{\mu_\mathrm{e}}{{R_0}^3} & = {\omega_0}^2 \qquad (\omega_0 \text{: 軌道角速度}) \\
\bm{g}^\mathrm{b} & \fallingdotseq 3 {\omega_0}^2
\begin{bmatrix}
(I_{zz} - I_{yy}) \phi \\
(I_{zz} - I_{xx}) \theta \\
0
\end{bmatrix}
\end{align}
どういう形がよい?
\begin{tikzpicture}%%角度足す
\coordinate (O) at (0, 0);

\begin{scope}[-latex]
\draw (O)--(-1, -2)node[below]{$z$};
\draw (O)--(0, -2.2)node[below]{$z_0$(地心)};
\draw (O)--(2, 0)node[right]{$y_0$};
\draw (O)--(2, -0.5)node[right]{$y$};
\end{scope}

\draw (O) circle [radius = 0.3];
\node [above left = 1] at (O) {$x_0$};
\end{tikzpicture}

\begin{tikzpicture}%%	円柱
\coordinate (O) at (0, 0);

\begin{scope}[-latex, thick]
\draw (O)--(2, -2)node[below right]{$x$};
\draw (O)--(2.4, 2)node[above right]{$y$};
\draw (O)--(0, 3)node[above]{$z$};
\end{scope}

\draw (-2, 1)--(-2, -1);
\draw (2, 1)--(2, -1);
\draw (0, 1) circle [x radius = 2, y radius = 0.8];
\draw (0, -1) circle [x radius = 2, y radius = 0.8];
\end{tikzpicture}


\begin{itemize}
\item $I_{zz}-I_{yy}>0$\\
$\phi$が正の時トルクも正$\rightarrow$
ずれが拡大$\rightarrow$
不安定

\begin{tikzpicture}%%円柱
\coordinate (O) at (0, 0);
\begin{scope}[->]
\draw(O)--(0, 2);
\draw(O)--(0, -2);
\end{scope}

\draw (1, 0.5)--(1, -0.5);
\draw (-1, 0.5)--(-1, -0.5);
\draw (0, 0.5) circle [x radius = 1, y radius = 0.4];
\draw (0, -0.5) circle [x radius = 1, y radius = 0.4];
\end{tikzpicture}

\item  $I_{zz}-I_{yy}<0$\\
$\phi$が正の時トルクは負$\rightarrow$
ずれが小さくなる$\rightarrow$
安定
\end{itemize}
軌道上への「置き方」が重要\\
同様に$\theta$についての安定条件
$=I_{zz}-I_{xx}<0$\\
\qquad$\rightarrow$
「$I_{zz}$が最小」が安定のための必要条件\\
(補足)
\begin{enumerate}
\item gravity gradient torqueは外乱であると同時に,受動的な
姿勢制御に使える($1^\circ\sim2^\circ$)
\item 安定といっても$\theta,\phi \rightarrow0$
に収束しない(0周りに振り子運動する=liberation)
\end{enumerate}
$<$姿勢運動(地球指向のため)も考慮に入れたとき$>$\\
orbital frame 自体が動く(動座標系)\\
基準座標系としてorbital frameをとる($X,Y,Z$)$F_0$
($F_i\cdots$慣性系)\\
$\rightarrow$衛星とともにY軸周りに$\omega_0$で
回転\\
(X,Y,Z)$\rightarrow$xyz(b-frame)への変換を
$\underbrace{F_p}_{\mathrm{principle}\quad\mathrm{axis}},
(\phi,\theta,\psi)$の微小回転で表現する.\\
やりたいこと
\begin{equation}
\text{Eulerの式}
\bm{M}^p=
I(\dot{\bm{\omega}}^p)+
\bm{\omega}^p\times
(I\bm{\omega}^p)
\qquad
\text{を書き下したい}
\end{equation}



$\bm{\omega}$ : $F_p$の$F_i$に対する角速度
\begin{align}
= \underexplline{\bm{\omega}_0}{i \rightarrow O} + \underexplline{\bm{\omega}_\alpha}{O \rightarrow p}, 
&
{\bm{\omega}_o}^{i \rightarrow O} =
\begin{bmatrix}
0 \\ -\omega_C \\ 0
\end{bmatrix}
\left( \omega_C = \sqrt{\frac{\mu}{R^3}} \right) \\
 & {\bm{\omega}_\alpha}^{p} = 
\begin{bmatrix}
\dot{\phi} \\ \dot{\theta} \\ \dot{\psi}
\end{bmatrix}
\end{align}
$F_0$から$F_p$へのDCMは近似的に,
\begin{equation}
C_0^p = 
\begin{bmatrix}
1 & \psi & -\theta \\
\psi & 1 & \phi \\
\theta & -\phi & 1
\end{bmatrix}
\qquad \qty(\theta, \ \phi, \ \psi \text{が十分小さい})
\end{equation}
\begin{align}
\rightarrow \bm{\omega}^p & = C_0^p \bm{\omega}_0^o + \bm{\omega}_o^p \\
& = C_o^p \begin{bmatrix} 0 \\ -\omega_C \\ 0 \end{bmatrix} + \begin{bmatrix} \dot{\phi} \\ \dot{\theta} \\ \dot{\psi} \\
& = 
\begin{bmatrix}
\dot{\phi} - \omega_C \psi \\
\dot{\theta} - \omega_C \\
\dot{\psi} + \omega_C \phi
\end{bmatrix}
\end{align}
ここで,2行目$\rightarrow$3行目でEulerの式
\begin{align}
I_{xx} \dot{\omega_x} + (I_{zz} - I_{yy})\omega_y \omega_z = 3 {\omega_C}^2 (I_{zz} - I_{yz})\phi \\
I_{yy} \dot{\omega_y} + (I_{xx} - I_{zz})\omega_x \omega_z = 3{\omega_C}^2 (I_{zz} - I_{xx}) \theta \\
I_{zz} \dot{\omega_z} + (I_{yy} - I_{zz}) \omega_y \omega_z = 0
\end{align}
簡略化のために
\begin{equation}
I_{xx}\equiv I_1,
I_{yy}\equiv I_2,
I_{zz}\equiv I_3\quad
\text{とする.}
\end{equation}
(Eulerの式の一つ目)
\begin{align}
I_1\ddot{\phi}-
I_1\omega_c\dot{\psi}+
(I_3-I_2)(-\omega_c+\dot{\theta})
(\oemga_c\phi+\dot{\psi})\\
=I_1\ddot{\phi}-(I_1+I_3-I_2)
\omega_c\dot{\psi}+
(I_2-I_3)\omega_c-2\phi+
\mathcal{O}(\epsilon^2)=
3\oemga_c^2(I_3-I_2)\phi
\end{align}
外乱がgravity gradientのみの時に
\begin{align}
\left{
I_1\ddot{\theta}-(I_1+I_3-I_2)\omega_c\dot{\pho}+
4(I_2-I_3)\omega_c-2\phi&
=0 \label{eq:1}\\
I_2\ddot{\theta}+3(I_1-I_3)\omega_c^2\theta
&=0 \label{eq:2}\\
I_3\ddot{psi}
+(I_1+I_3-I_2)\omega_c\dot{\psi}
+(I_2-I_1)\omega_c^2\psi
&=0\label{eq:3}
\right
\end{align}
定性的に
\begin{enumerate}
\item y軸まわりは独立している.$I_1 > I_3$が安定のための必要十分条件
\item x,z軸は$\omega_c$を介してカップリングしている.
(1),(3)より\\
A = 
\begin{pmatrix}
I_1 & 0\\
0 & I_3 \\
\end{pmatrix}
B = \begin{pmatrix}
0 & -1\\
1 & 0\\
\end{pmatrix}
$(I_1 + I_3 - I_2) \omega_c$
C = 
\begin{pmatrix}
4(I_2- I_3) & 0\\
0 & I_2 - I_1\\
\end{pmatrix}
\begin{equation}
\bm{q} = 
\begin{pmatrix}
\phi \\
\psi \\
\end{pmatrix}
\end{equation}
\begin{equation}
A \ddot{\bm{q}} + B \dot{\bm{q}} + C \bm{q} = \bm{0} 
\end{equation}
安定条件は?
\end{enumerate}
\undeline{十分条件} Cがpositive definite(正定)であること
$(\forall x, \bm{x}^T C \bm{x} > 0)$ \\
\therefore $I_2 - I_3 >0$かつ$I_2 - I_1 > 0$\\
\rightarrow 1,2から安定のための十分条件は
\begin{equation}
\overbrace{I_2}^{\text{ピッチ}} 
>
\overbrace{I_1}^{\text{ろー}} > I_3
(\text{\omega_c を考えなければ}I_2 > I_3 , I_1 > I_3)
$<$補足$>$もし厳密な必要十分条件を出すなら
\begin{equation}
k_1=
\frac{I_2-I_3}{I_1},
k_3=
\frac{I_2-I_1}{I_3
}\qquad\text{で運動方程式を書き換えると}
\end{equation}
\begin{equation}
s^4+
(1+3k_1+k_1k_3)\omega_c^2s^2+
4k_1k_3\omega{c}^4=0
\text{(特性方程式)}
\end{equation}
4つの根が 左半面にある条件
\begin{itemize}
\item $1+3k_1+k_1k_3>0$
\item $k_1k_3>0$
\item $
(1+3k_1+k_1k_3)^2-
16k_1k_3>0
$
\end{itemize}
\end{document}
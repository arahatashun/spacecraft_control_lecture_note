\documentclass[class=article, crop=false, dvipdfmx, fleqn]{standalone}
%% preamble

% titles
\title{宇宙機制御工学}
\date{\the\year/\the\month/\the\day}
\author{Shun Arahata \and Hirotaka Kondo \and Ryosuke Morita}

% packages and libraries
\usepackage[utf8]{inputenc}				%fonts
\usepackage[ipaex]{pxchfon}
\usepackage{pifont}
\usepackage{mathtools, amssymb, mathrsfs, bbm}	%math
\usepackage{siunitx, physics}
\usepackage[table]{xcolor}				%colors
\usepackage{graphicx}					%figures
\usepackage{subcaption, wrapfig}
\usepackage{tikz}
\usetikzlibrary{calc, patterns, decorations, angles, calendar, backgrounds, shadows, mindmap, decorations.pathreplacing}
\usepackage{tcolorbox}					%tables
\usepackage{longtable, float, multirow, array, listliketab, enumitem, tabularx}
\usepackage{listings}					%listings
\usepackage{comment}
\usepackage{hyperref}					%URL, link
\usepackage{url}
\usepackage{pxjahyper}
\usepackage{overcite}					%setting of citation
\usepackage{pxrubrica}					%rubi
\usepackage{fancyhdr, lastpage}			%pagelayout
\usepackage{import, grffile}			%file management
\usepackage{standalone}
\usepackage{bm}
\usepackage{empheq}
% set up for hyperref
\hypersetup{%
	bookmarksnumbered = true,%
	hidelinks,%
	colorlinks = true,%
	linkcolor = black,%
	urlcolor = cyan,%
	citecolor = black,%
	filecolor = magenta,%
	setpagesize = false,%
	pdftitle = {宇宙機制御工学},%
	pdfauthor = {Shun Arahata, Hirotaka Kondo, Ryosuke Morita},%
	pdfkeywords = {},%
	}

% set up for siunitx
\sisetup{%
	%detect-family = true,
	detect-inline-family = math,
	detect-weight = true,
	detect-inline-weight = math,
    %input-product = *,
    quotient-mode = fraction,
	fraction-function = \frac,
	inter-unit-product = \ensuremath{\hspace{-1.5pt}\cdot\hspace{-1.5pt}},
	per-mode = symbol,
	product-units = single,
	}

% setting of line skip
\setlength{\lineskiplimit}{6pt}
\setlength{\lineskip}{6pt}

% setting of indent
\setlength{\parindent}{1zw}
\setlength{\mathindent}{5zw}

% change cite form
\renewcommand{\citeform}[1]{[#1]}

% number equations only when they are referred to in the text
\mathtoolsset{showonlyrefs=true}

% decrement for section
\setcounter{section}{-1}

% circle numbers
\def\maru#1{%
	\leavevmode \setbox0\hbox{$\bigcirc$}%
	\copy0\kern-\wd0 \hbox to\wd0{\hfil{\small#1}\hfil}%
    }

% add explanation to formulas (needs debug)
\newcommand{\overexpl}[3][2pt]{%
	\overset{\hspace{-105mm}\rule[-#1]{0pt}{#1}#3\hspace{-105mm}}{\overbrace{#2}}%
    }
\newcommand{\underexpl}[3][0pt]{%
	\underset{\hspace{-105mm}\rule[0pt]{0pt}{#1}#3\hspace{-105mm}}{\underbrace{#2}}%
    }
\newcommand{\underexplline}[3][0pt]{%
	\underset{\hspace{-105mm}\rule[0pt]{0pt}{#1}#3\hspace{-105mm}}{\underline{#2}}%
    }

% define color of emphasize
\definecolor{midnightblue}{RGB}{25, 105, 112}
\definecolor{mediumblue}{RGB}{0, 0, 165}
\definecolor{royalblue}{RGB}{65, 25, 225}
\definecolor{limegreen}{HTML}{32CD32}
\definecolor{darkgreen}{HTML}{006400}
\definecolor{forestgreen}{HTML}{228B22}
\definecolor{orangered}{HTML}{FF4500}

\definecolor{emph}{HTML}{FF4500}

\begin{document}
\section{衛星(Rigid Body)のDynamics}
\begin{center}
トルク
$\xrightarrow[\text{Dynamics}]{}$
$\underline{\bm{\omega}}$
$\xrightarrow[\text{Kinematics}]{}$
姿勢
\end{center}



\begin{tikzpicture}
\coordinate (O) at (0, 0);
\coordinate (C) at (2, 1);
\coordinate (R) at (2.2, 2);

\begin{scope}[-latex]	%axis
\draw (O)--+(-1, -1)node[below left]{$X$}node[above = 2]{$\bm{I}$};
\draw (O)--+(3, 0)node[right]{$Y$}node[below left = 1]{$\bm{J}$};
\draw (O)--+(0, 3)node[above]{$Z$}node[below left = 1]{$K$};
\end{scope}

\begin{scope}[-stealth, thick]	%vecs
\draw (O)--(C)node[below]{$C$};
\draw (O)--(R);
\draw (C)--(R)node[below right]{$\bm{r}$};
\end{scope}
\draw (0.6, 0.3) to[bend left] (0.8, -0.4)node[below]{$\bm{R}_c$};

\begin{scope}[-stealth, emph, thick]	%coloemph vecs
\draw (C)--+(0.7, 0.35)node[below = 1]{$x(\bm{i})$};
\draw (C)--+(0.4, 0.6)node[right]{$y(\bm{j})$};
\draw (C)--+(-0.4, 0.8)node[left]{$z(\bm{k})$};
\end{scope}

\node at ($(O)!0.3!(R) + (-0.2, 0.3)$) {$\bm{R}$};
\node [above right] at (R) {$\bm{P}(dm)$};

\end{tikzpicture}

\begin{align}
XYZ\left(\bm{I},\bm{J},\bm{K}\right)&:
\text{慣性座標系}\\
xyz\left(\bm{i},\bm{j},\bm{k}\right)&:
\text{動座標系}\\
\text{C}&:
\text{動座標系の原点}\\
\bm{P}&:
\text{微小質量dmの位置}
\end{align}
基本式
\begin{align}
\bm{\dot{x}} = 
\bm{\dot{x}_{rel}} + 
\bm{\omega}\times\bm{x}
\end{align}
$\bm{\dot{x}_{rel}}$は「動座標系から見た$\bm{x}$」の変化率\\
質点$dm$の位置,速度,加速度は
\begin{align}
\bm{R}&=
\bm{R}_c+\bm{r}\\
\bm{V}&=
\dot{\bm{R}}_c+\dot{\bm{r}}\\
&=
\bm{V}_c+\bm{V}_\mathrm{rel}+
\bm{\omega}\times\bm{r}\qquad
%(\bm{\omega}\times\bm{R}_c = \bm{0})
%おかしくない?
\\
\bm{a}
&=
\dot{\bm{V}}_c+
\dot{\bm{V}}_\mathrm{rel}+
\dot{\bm{\omega}}\times\bm{r}+
\bm{\omega}\times\dot{\bm{r}}\\
&=
\bm{a}_c+
\left(
\bm{a}_\mathrm{rel}
+\bm{\omega}\times\bm{V}_\mathrm{rel}
\right)
+\dot{\bm{\omega}}
\times \bm{r}
+\bm{\omega}\times
\left(
\bm{V}_\mathrm{rel}
+\bm{\omega}\times\bm{r}
\right)\\
&=
\bm{a}_c+
\bm{a}_{rel}+
\underbrace{2\bm{\omega}\times\bm{V}_{rel}}
_{\text{コリオリ力}}+
\dot{\bm{\omega}}\times\bm{r}+
\underbrace{\bm{\omega}\times
\left( \bm{\omega} \times 
\bm{r} \right)}_{\text{遠心力}}
\end{align}


\begin{itemize}
\item 以降,成分表記はb-frameから見たものとする
\item $dm$の位置$\bm{r}$はb-frameに固定されている
(剛体)$\rightarrow \bm{V}_\mathrm{rel} = \bm{0}, \bm{a}_\mathrm{rel}
= \bm{0}$
\end{itemize}


\subsection{諸量の表現}
\begin{enumerate}[label = \maru{\theenumi}]
\item{運動量(Linear Momentum)}
\begin{align}
\bm{P}
\equiv \int_m(\bm{v})dm 
& = \int_m
(\bm{V}_c +
\underbrace{\bm{V}_\mathrm{rel}}_{\bm{0}} +
\bm{\omega} \times \bm{r})dm \\
& = 
\bm{V}_c \int_mdm +
\bm{\omega}\times \underbrace{\int_m\bm{r}dm}_{\bm{0}}\\
\intertext{原点を重心に選ぶと$\int_m\bm{r}dm = \bm{0}$であるから}
& = M\bm{V}_c (\text{M:全体の質量}) 
\end{align}

\item 角運動量(Angular Momentum)(C点周りの)
\begin{equation}
 \bm{H}_c 
 \equiv
 \int_m \bm{r} \times \bm{V} dm 
 =\int_m \bm{r} \times (\bm{V}_c + \bm{\omega} \times \bm{r}) dm
 \end{equation}
ここで,
\begin{equation}
\int_m \bm{r} \times \bm{V}_c \ dm 
= \int_m \bm{r} \ dm \times \bm{V}_c 
= \bm{0}
\end{equation}
また,
$\bm{r} = \begin{pmatrix} x \\ y \\ z \end{pmatrix}$,
$\bm{\omega} = \begin{pmatrix} \omega_x \\ \omega_y \\ \omega_z \end{pmatrix}$
とおくと,
\begin{equation}
\int_m \bm{r} \times (\bm{\omega} \times \bm{r}) dm = \int_m \bm{r} \times
\begin{pmatrix}
\omega_y z - \omega_z y \\
\omega_z x - \omega_x z \\
\omega_x y - \omega_y x \\
\end{pmatrix}
dm
=
\int_m 
\begin{bmatrix}
\omega_x (y^2 + z^2) - \omega_y xy - \omega_z xz \\
\omega_y (x^2 + z^2) - \omega_z yz - \omega_x xy \\
\omega_z (x^2 + y^2) - \omega_x xz - \omega_y yz
\end{bmatrix}
dm
\end{equation}
\begin{align}
\int_m (x^2 + z^2) dm = I_{yy}, &
\int_m (y^2 + z^2) dm = I_{xx}, &
\int_m (x^2 + y^2) dm = I_{zz}
\end{align}
\begin{align}
\int_m xy dm = I_{xy}, \dots \text{etc}
\qquad\text{慣性乗積}
\end{align}
\begin{align}
\rightarrow
\bm{H}_c^b & =
\begin{bmatrix}
I_{xx} \omega_x 
- I_{xy}\omega_y
- I_{xz}\omega_z \\
I_{yy} \omega_y
-I_{xy}\omega_x
- I_{yz}\omega_z \\
I_{zz} \omega_z 
- I_{xz}\omega_x 
- I_{yz}\omega_y
\end{bmatrix}
=
\underbrace
{
\begin{bmatrix}
I_{xx} &
\overexpl{-I_{xy}}{\textcolor{emph}{\text{符号に注意}}} &
-I_{xz} \\
-I_{xy} & I_{yy} & -I_{yz} \\
-I_{xz} & -I_{yz} & I_{zz}
\end{bmatrix}
}_
{\text{慣性テンソル}}
\begin{bmatrix}
\omega_x \\
\omega_y \\
\omega_z
\end{bmatrix} \\
& = [I_\mathrm{ver}] \bm{\omega}^b
\end{align}

\item 力学的エネルギー(kinetic energy)
\begin{align}
T
& =
\frac{1}{2}\int_m \bm{V} \cdot \bm{V} dm
= 
\frac{1}{2} \int_m
\left( \bm{V}_c + \bm{\omega} \times \bm{r} \right)\cdot
\left( \bm{V}_c + \bm{\omega} \times \bm{r} \right)dm\\
& = 
\frac{1}{2} \int_m V_c^2 dm
+ 2 \cdot \frac{1}{2} \bm{V}_c \cdot
\int_m
\underexpl[10pt]{\bm{\omega} \times \bm{r}}{\bm{\omega}\text{を外に出せる}} \ dm
+\frac{1}{2}\int_m
|\bm{\omega}\times\bm{r}|^2dm
\\
&=\frac{1}{2}MV_c^2
+ \frac{1}{2}
\int_m
\left\{
\left(
\omega_{y}z
-\omega_{z}y
\right)^2
+
\left(
\omega_{z}x
-\omega_{x}z
\right)^2
+
\left(
\omega_{x}y
-\omega_{y}x
\right)^2
\right
\}
dm\\
& = 
\frac{1}{2} M {V_c}^2 +
\frac{1}{2}
\left[
I_{xx}{\omega_x}^2 + 
I_{yy}{\omega_y}^2 +
I_{zz}{\omega_z}^2 -
2 \omega_y \omega_z I_{yz}-
2 \omega_z \omega_x I_{xz} -
2 \omega_x \omega_y I_{xy}
\right] \\
&=
\underset{\large\rule[0pt]{0pt}{10pt}\text{\textcolor{emph}{並進}}}{\frac{1}{2} M{V_c}^2}
\underset{\large\rule[0pt]{0pt}{16pt}\textcolor{emph}{\leftrightarrow}}{+}
\underset{\large\rule[0pt]{0pt}{10pt}\hspace{-20mm}\text{\textcolor{emph}{回転}}}{%
\frac{1}{2}
\begin{bmatrix}
\omega_x & \omega_y & \omega_z
\end{bmatrix}
{\bm{H}_c}^b
}
\end{align}
\end{enumerate}


\subsection{Equation of Motion}
\begin{align}
\bm{F} = \bm{\dot{P}}(\dot{\text{並進の運動量}}) 
         = M\bm{\dot{V}_c}
\end{align}
\begin{align}
 \bm{M}_c &= \bm{\dot{H}}_c
(\dot{\text{c点周りの角運動量}}) \\
 &= \underexpl{\bm{\dot{H}}_{c,rel}}{\bm{H}_c^b\text{の時間微分}} +
 \bm{\omega} \times \bm{H}_c
 \end{align}
 \begin{align}
\rightarrow \bm{M}^b_c = 
[Iner]
\begin{bmatrix}
   \dot{\omega}_x\\
   \dot{\omega}_y\\
   \dot{\omega}_z\\
\end{bmatrix}
+
\begin{bmatrix}
   {\omega}_x\\
   {\omega}_y\\
   {\omega}_z\\
\end{bmatrix}
\times
\left(
[Iner]
\begin{bmatrix}
 {\omega}_x\\
   {\omega}_y\\
   {\omega}_z\\
 \end{bmatrix}
\right)
\end{align}
$x,y,z$を慣性主軸に取ると
\begin{equation}
I_{xy} = I_{yz} = I_{zx} = 0
\qquad\rightarrow\qquad
[Iner] = 
\begin{bmatrix}
  I_{xx} & 0 & 0 \\
  0 & I_{yy} & 0 \\
  0 & 0 & I_{zz} 
\end{bmatrix}
\end{equation}
\begin{equation}
\rightarrow\bm{M}_c^b
=\begin{bmatrix}
M_x\\
M_y\\
M_z
\end{bmatrix}
=\begin{bmatrix}
I_{xx}\dot{\omega}_x
+(I_{zz}-I_{yy})\overbrace{\omega_y\omega_z}^{\textcolor{emph}{\text{非線形項}}}\\
I_{yy}\dot{\omega_y}+
(I_{xx}-I_{zz})\omega_x\omega_z\\
I_{zz}\dot{\omega_z}+
(I_{yy}-I_{xx})\omega_x\omega_y
\end{bmatrix}
\qquad
\text{Eulerの式}
\end{equation}



\begin{comment}
\newpage
色サンプル \par
\textcolor{mediumblue}{強調}\par
\textcolor{midnightblue}{強調}\par
\textcolor{royalblue}{強調}\par
\textcolor{limegreen}{強調}\par
\textcolor{darkgreen}{強調}\par
\textcolor{forestgreen}{強調}\par
\textcolor{orangered}{強調}\par
等々
\end{comment}


\end{document}
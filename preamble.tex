%% preamble

% titles
\title{宇宙機制御工学}
\date{\the\year/\the\month/\the\day}
\author{Shun Arahata\and Hirotaka Kondo \and Ryosuke Morita}

% packages and librariees
\usepackage[utf8]{inputenc}				%Japanese fonts
\usepackage[ipaex]{pxchfon}
\usepackage{amsmath,amssymb, mathrsfs}	%math
\usepackage{siunitx, physics}
\usepackage{graphicx}					%figures
\usepackage{subcaption, wrapfig}
\usepackage{tikz}
\usetikzlibrary{calc, patterns, decorations, angles, calendar, backgrounds, shadows, mindmap}
\usepackage{tcolorbox}					%tables
\usepackage{longtable, float, multirow, array, listliketab, enumitem, tabularx}
\usepackage{listings}					%listings
\usepackage{comment}
\usepackage{hyperref}					%URL, link
\usepackage{url}
\usepackage{pxjahyper}
\usepackage{overcite}					%setting of citation
\usepackage{pxrubrica}					%rubi
\usepackage{fancyhdr, lastpage}			%pagelayout
\usepackage{import, grffile}			%file management
\usepackage{standalone}

% set up for hyperref
\hypersetup{%
	bookmarksnumbered = true,%
	hidelinks,%
	colorlinks = true,%
	linkcolor = black,%
	urlcolor = cyan,%
	citecolor = black,%
	filecolor = magenta,%
	setpagesize = false,%
	pdftitle = {宇宙機制御工学},%
	pdfauthor = {Shun Arahata, Hirotaka Kondo, Ryosuke Morita},%
	pdfkeywords = {},%
	}

% set up for siunitx
\sisetup{%
	%detect-family = true,
	detect-inline-family = math,
	detect-weight = true,
	detect-inline-weight = math,
    %input-product = *,
    quotient-mode = fraction,
	fraction-function = \frac,
	inter-unit-product = \ensuremath{\hspace{-1.5pt}\cdot\hspace{-1.5pt}},
	per-mode = symbol,
	product-units = single,
	}

% setting of line skip
\setlength{\lineskiplimit}{6pt}
\setlength{\lineskip}{6pt}

% change cite form 
\renewcommand{\citeform}[1]{[#1]}